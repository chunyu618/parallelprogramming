\documentclass{beamer}
\usetheme{Warsaw}

\usepackage{color}
\usepackage{listings}
\usepackage{url}
\usepackage{pgf}


% global setting

\newcommand{\comment}[1]{}

% list environment gloabal setting

% to specify the size of fonts in lst environment

\lstset{showstringspaces=false}
\lstset{language=C}
\lstset{basicstyle=\tt}
\lstset{keywordstyle=\color{blue}}
\lstset{emphstyle=\color{red}}
\lstset{numbers=left}
\lstset{frame=single}
\lstset{rangeprefix=\/*\ }
\lstset{rangesuffix=\ *\/}
\lstset{includerangemarker=false}

%% slides related macros

\newenvironment{mytitle}[2]
{\begin{frame}{#1} \vspace{0.2\textheight}{\huge #2} \end{frame}}

%% Macro for non-numbered frames listing 
%% \nonumberlisting{title}{file}{options}

\newcommand{\nonumberlisting}[3]
{\noindent{#1}
{\lstinputlisting[#3]{#2}}}

%% Program listing macros

%% \programlisting{file}{desp}{font}
%% \programlistingfirst{file}{desp}{font}{start}{end}
%% \programlistingrest{file}{descp}{font}{start}{end}


%% List a program in one page with a font size.

\newcommand{\programlisting}[3]
{\renewcommand{\lstlistingname}{#3 \bf Example}
\lstinputlisting[basicstyle={#3 \tt},caption={#3 \bf (#1) #2}]{#1}}

%% List a program for the first page within a given range.

\newcommand{\programlistingfirst}[5]
{\renewcommand{\lstlistingname}{#3 \bf Example}
\lstinputlisting[basicstyle={#3 \tt},linerange={#4-#5},caption={#3 \bf (#1) #2}]{#1}}

%% List a program for the second page (or later).

\newcommand{\programlistingrest}[5]
{\lstinputlisting[basicstyle={#3 \tt},linerange={#4-#5}]{#1}}

%% List a program in two pages.

\newcommand{\programlistingtwoslides}[6]
{\begin{frame}                  % do not add [fragile]
\programlistingfirst{#1}{#2}{#3}{#4}{#5}
\end{frame}
\begin{frame}
\programlistingrest{#1}{#2}{#3}{#5}{#6}
\end{frame}}

%% List a program in three pages.

\newcommand{\programlistingthreeslides}[7]
{\begin{frame}                  % do not add [fragile]
\programlistingfirst{#1}{#2}{#3}{#4}{#5}
\end{frame}
\begin{frame}
\programlistingrest{#1}{#2}{#3}{#5}{#6}
\end{frame}
\begin{frame}
\programlistingrest{#1}{#2}{#3}{#6}{#7}
\end{frame}
}


%% Program input/output macros

%% \programoutput{file}
%% \programoutputfirst{file}{font}{options}
%% \programoutputrest{file}{font}{startline}{endline}
%% \programoutputtwoslides{file}{font}{start}{endl}{start2}{end2}

%% Program output macor with default setting

\newcommand{\programoutput}[1]
{\programoutputfirst{#1}{}{}}

%% Program output macor for the first page

\newcommand{\programoutputfirst}[3]
{\lstset{basicstyle={#2 \tt}}
\nonumberlisting{#2 \bf 輸出}{#1.out}{#3}}

%% Program output macor for later page

\newcommand{\programoutputrest}[4]
{\lstinputlisting[firstnumber={#3},basicstyle={\tt #2},linerange={#3-#4}]
{#1.out}}

%% Program output macor for two slides

\newcommand{\programoutputtwoslides}[6]
{\begin{frame}                  % do not add [fragile]
\programoutputfirst{#1}{#2}{linerange={#3-#4}}
\end{frame}
\begin{frame}
\programoutputrest{#1}{#2}{#5}{#6}
\end{frame}}


%% \programinput{file}
%% \programinputfirst{file}{font}{option}

%% Program input macor with default setting

\newcommand{\programinput}[1]
{\programinputfirst{#1}{}{}}

%% Program input macor for the first page

\newcommand{\programinputfirst}[3]
{\lstset{basicstyle={#2 \tt}}
\nonumberlisting{#2 \bf 輸入}{#1.in}{#3}}


%% \piotwocol{file}

\newcommand{\piotwocol}[1]{\piotwocoldetail{#1}{}{}}

\newcommand{\piotwocoldetail}[2]{
\begin{columns}[t,onlytextwidth]
\begin{column}{0.45\textwidth}
\programinputfirst{#1}{#2}{}
\end{column}
\begin{column}{0.45\textwidth}
\programoutputfirst{#1}{#2}{}
\end{column}
\end{columns}}

%% \piotworow{file}

\newcommand{\piotworow}[1]
{\programinput{#1}
\programoutput{#1}}

\newcommand{\piotworowdetail}[2]
{\programinputfirst{#1}{#2}{}
\programoutputfirst{#1}{#2}{}}


%% \newcommand{\headerfile}[1]
%% {\nonumberlisting {{\bf 標頭檔} {\tt #1}}{#1}{}}

\newcommand{\header}[2]{\headerdetail{#1}{#2}{}}

\newcommand{\headerdetail}[3]
{{\renewcommand{\lstlistingname}{#3 \bf Header}
\lstset{basicstyle={#3 \tt}}
\lstset{label=header:#1,caption={(#1) #2}}
\lstinputlisting{#1}}}

%% Macro for prototype
%% \prototype{file}

\newcommand{\prototype}[1]{\prototypedetail{#1}{}}

%% Macro for prototype with detialed setting
%% \prototypedetail{file}{font}

\newcommand{\prototypedetail}[2]
{{\renewcommand{\lstlistingname}{#2 \bf Prototype}
\lstset{basicstyle={#2 \tt}}
\lstset{frame=single,label=prototype:#1,caption={#2}}
\lstinputlisting{#1}}}

%% Macro for phrase
%% \nphrase{file}{desc}

%% \newcommand{\nphrase}[2]
%% {{\renewcommand{\lstlistingname}{\lstfontsize \bf 片語}
%% \lstset{label=phr:#1,frame=single,caption={\lstfontsize #2}}
%% \lstset{basicstyle={\lstfontsize \tt}}
%% \lstinputlisting{#1.p}}
%% }

\newcommand{\nphrase}[2]
{\nphrasedetail{#1}{#2}{}}

%% Macro for detail phrase
%% \nphrasedetail{file}{desc}{font}

\newcommand{\nphrasedetail}[3]
{{\renewcommand{\lstlistingname}{#3 \bf 片語}
\lstset{label=phr:#1,frame=single,caption={#3 #2}}
\lstset{basicstyle={#3 \tt}}
\lstinputlisting{#1.p}}}



\newcommand{\inputfile}[1]
{\nonumberlisting {{\bf 輸入檔} {\tt #1}}{#1}{}}

\newcommand{\outputfile}[1]
{\nonumberlisting {{\bf 輸出檔} {\tt #1}}{#1}{}}

% Macro for listings that are labeled.
% [1] is the filename, [2] is the description

\newcommand{\commandline}[1]
{\lstset{basicstyle={\tt}}
\nonumberlisting{\bf 命令列}{#1.bat}{}}

% {\boxedlisting{\bf 命令列}{#1.bat}}

% Macro for listings that are labeled.
% [1] is the filename, [2] is the description

\newcommand{\out}[2]
{{\renewcommand{\lstlistingname}{\bf 執行結果}
\lstset{label=out:#1,caption=#2}
\lstinputlisting{#1.out}}}






\begin{document}

\title{Intermediate OpenCL Programming}

\author{Pangfeng Liu \\ National Taiwan University}

\begin{frame}
  \titlepage
\end{frame}

\section{Buffer Properties}

\begin{frame}
  \frametitle{Buffer} 
  There are three exclusive types of buffers.  
  \begin{description}
  \item [CL\_MEM\_USE\_HOST\_PTR]
  \item [CL\_MEM\_ALLOC\_HOST\_PTR]
  \item [CL\_MEM\_COPY\_HOST\_PTR]
  \end{description}
\end{frame}

\begin{frame}
  \frametitle{CL\_MEM\_USE\_HOST\_PTR} 
  \begin{itemize}
  \item OpenCL uses the host memory as the buffer so the {\tt
    host\_ptr} must not be {\tt NULL}.
  \item If the device is GPU then the access is slow because the
    buffer is in the host (CPU).
  \end{itemize}
\end{frame}

\begin{frame}
  \frametitle{CL\_MEM\_ALLOC\_HOST\_PTR} 
  \begin{itemize}
  \item OpenCL will allocate a buffer in the device and the device
    will use it directly.
  \item This memory is accessible from the host by {\tt
    clEnqueueWriteBuffer} and {\tt clEnqueueReadBuffer}.
  \item Since the device will allocate the buffer we cannot provide a
    non-NULL {\tt host\_ptr}.
  \item The access is fast because the buffer is in the device
  \end{itemize}
\end{frame}

\begin{frame}
  \frametitle{CL\_MEM\_COPY\_HOST\_PTR} 
  \begin{itemize}
  \item OpenCL will copy the contents of the host buffer to a
    buffer on the device
  \item The device will use the device buffer.
  \item The access is fast because the buffer is in the device
    (GPU/CPU).
  \end{itemize}
\end{frame}

\begin{frame}
  \frametitle{Comparison}
  \begin{tabular}{|l|l|l|l|} \hline
    buffer type & {\tt host\_ptr} & device will use & speed from GPU \\ \hline
    {\tt USE} & {\tt != NULL} & the host buffer directly & slow \\ \hline
    {\tt ALLOC} & {\tt == NULL} & the device buffer & fast \\ \hline
    {\tt COPY} & {\tt != NULL} & the device buffer & fast \\
    & & copied from the &  \\ 
    & & host buffer & \\ \hline
  \end{tabular}
\end{frame}

\begin{frame}
  \frametitle{Discussion}
  \begin{itemize}
    \item Which of the allcoation mode provides the slowest memory
      access from a device GPU?
  \end{itemize}
\end{frame}

\begin{frame}
  \frametitle{Need not to Retrieve}
  \begin{itemize}
    \item We would like to implement the vector addition program
      without getting the  results back from the device.
    \item The idea is that we copy the vector A and B into device with
      {\tt CL\_MEM\_COPY\_HOST\_PTR} buffer, and put the results in a
      {\tt CL\_MEM\_USE\_HOST\_PTR} buffer.
    \item When the computation is over then we can get the results in
      the host buffer directly.
  \end{itemize}
\end{frame}

\begin{frame}
  \programlistingfirst{vectorAdd-nofetchC.c}{}{\scriptsize}{createbuffer}{setarg}
\end{frame}

\begin{frame}
  \frametitle{Execution}
  \begin{itemize}
    \item After creating buffers, setting the parameter order, and
      set up the dimension of {\tt NDRange}, we run the kernel by
      placing it into the command queue
  \end{itemize}
\end{frame}

\begin{frame}
  \frametitle{\tt clFinish}
  \programlistingfirst{vectorAdd-nofetchC.c}{}{\footnotesize}{setshape}{checkandfree}
\end{frame}


\begin{frame}
  \frametitle{Demonstration}
  \begin{itemize}
    \item Run the {\tt vector-nofetchC-cl} program.

  \end{itemize}
\end{frame}

\begin{frame}
  \frametitle{Discussion}
  \begin{itemize}
  \item Does it produce the correct answer?
  \end{itemize}
\end{frame}

\begin{frame}
  \frametitle{Non-blocking}
  \begin{itemize}
    \item The previous program does not produce the correct answer
      because the function call {\tt clEnqueueNDRangeKernel} is {\em
        non-blocking}, which means it will not wait for the completion
      of the kernel.
    \item Since we check the contents of {\tt C} before the kernel
      finishes, the answer is incorrect.
    \item We need to wait for the commands in the command queue to
      finish.
  \end{itemize}
\end{frame}

\begin{frame}
  \prototypedetail{clFinish.h}{\footnotesize}
\end{frame}

\begin{frame}
  \frametitle{Parameters}
  \begin{description}
  \item [\tt command\_queue] Wait for the commands in this command
    queue to finish.
  \end{description}
\end{frame}

\begin{frame}
  \frametitle{\tt clFinish}
  \programlistingfirst{vectorAdd-nofetchC-finish.c}{}{\footnotesize}{setshape}{checkandfree}
\end{frame}

\begin{frame}
  \frametitle{Demonstration}
  \begin{itemize}
    \item Run the {\tt vector-nofetchC-finish-cl} program.
  \end{itemize}
\end{frame}

\begin{frame}
  \frametitle{Discussion}
  \begin{itemize}
  \item Does it produce the correct answer?
  \end{itemize}
\end{frame}

\section{Index Domain}

\subsection{Global Index}

\begin{frame}
  \frametitle{\tt NDRange} 
  \begin{itemize}
    \item We can set the dimension of {\tt NDRange} in the
      {\tt clEnqueueNDRangeKernel} call.
    \item In the previous example we set the dimension to one, now we
      want to set it to two.
    \item Now a work item in {\tt NDRange} will have two indices for
      two dimensions.
  \end{itemize}
\end{frame}

\begin{frame}
  \frametitle{Index}
  \begin{itemize}
    \item We can call {\tt get\_global\_id(i)} to know its index in
      the $i$-th dimension.
    \item The parameter {\tt i} must be within the dimension of {\tt
      NDRange}
  \end{itemize}
\end{frame}

\begin{frame}
  \prototypedetail{getGlobalId.h}{\scriptsize}
\end{frame}

\begin{frame}
  \frametitle{Parameters}
  \begin{description}
  \item [\tt dimindx] The dimension in which we want to know the
    index.
  \end{description}
\end{frame}

\begin{frame}
  \frametitle{Domain}
  \begin{itemize}
    \item We will use an array to keep track of the global indices a
      kernel function can see from a work item.
      \item We declare this array as {\tt int globalId[2][N][N]},
        where {\tt N} is 16.
      \item The first 2 is for two dimensions.
  \end{itemize}
\end{frame}

\begin{frame}
  \frametitle{Kernel}
  \programlisting{get-global-id.cl}{}{\footnotesize}
\end{frame}

\begin{frame}
  \frametitle{Kernel}
  \begin{itemize}
  \item We first call {\tt get\_global\_id(0)} and {\tt
    get\_global\_id(1)} to know the indices of this work item on the
    two dimensions, and put them into {\tt id0} and {\tt id1}.
  \item Then we place {\tt id0} and {\tt id1} into corresponding
        cells of {\tt globalId}.
  \end{itemize}
\end{frame}

\begin{frame}
  \frametitle{Discussion}
  \begin{itemize}
    \item Google {\tt \_\_global} to and find out what it means.
  \end{itemize}
\end{frame}

\begin{frame}
  \frametitle{Buffer}
  \begin{itemize}
    \item In the main program we declare a host buffer {\tt
      bufferGlobalId} to hold the global indices.
    \item This buffer will link to the {\tt globalId[2][N][N]}
      parameter in the kernel function.
  \end{itemize}
\end{frame}

\begin{frame}
  \frametitle{Buffers}
  \programlistingfirst{get-global-id.c}{}{\scriptsize}{createbuffer}{setshape}
\end{frame}

\begin{frame}
  \frametitle{Dimension}
  \begin{itemize}
    \item Now we set the dimension of {\tt NDRange} to 2, and set the
      size of each dimension to {\tt N}, as in {\tt globalDim}.
    \item Since {\tt NDRange} has two dimensions, a work group will
      also have two dimension.  For simplicity we set it to one by
      one, as in {\tt localDim}. 
      \item We place the kernel into the command queue, then call {\tt
        clFinish} to wait for the completion.
  \end{itemize}
\end{frame}

\begin{frame}
  \frametitle{Buffers}
  \programlistingfirst{get-global-id.c}{}{\scriptsize}{setshape}{printids}
\end{frame}

\begin{frame}
  \frametitle{printId}
  \begin{itemize}
    \item We implement a {\tt printId} function to print the contents
      in any array.
    \item The first parameter is a string for identification purpose,
      and the second parameter is the array of indices to print.
  \end{itemize}
\end{frame}

\begin{frame}
  \frametitle{Buffers}
  \programlistingfirst{get-global-id.c}{}{\scriptsize}{printid}{main}
\end{frame}

\begin{frame}
  \frametitle{Demonstration}
  \begin{itemize}
    \item Run the {\tt get-global-id-cl} program.
  \end{itemize}
\end{frame}

\begin{frame}
  \frametitle{Discussion}
  \begin{itemize}
  \item Describe the printed indices.
  \item Will the output be affected if we set the work group size
    differently?
  \end{itemize}
\end{frame}

\subsection{Local Index}

\begin{frame}
  \frametitle{\tt NDRange} 
  \begin{itemize}
    \item Local index is the index within a work group.
    \item We can specify the size of each dimension for a work group
      in a {\tt clEnqueueNDRangeKernel} call.
    \item In this example we will set the size of local dimension and
      observe the results.
  \end{itemize}
\end{frame}

\begin{frame}
  \frametitle{Local Index}
  \begin{itemize}
    \item We call {\tt get\_local\_id(i)} to know its index in the
      $i$-th dimension.
    \item The parameter {\tt i} must be within the dimension of {\tt
      NDRange} because the local and the global index have the same
      number of dimensions.
  \end{itemize}
\end{frame}

\begin{frame}
  \prototypedetail{getLocalId.h}{\scriptsize}
\end{frame}

\begin{frame}
  \frametitle{Parameters}
  \begin{description}
  \item [\tt dimindx] The dimension in which we want to know the
    index.
  \end{description}
\end{frame}

\begin{frame}
  \frametitle{Domain}
  \begin{itemize}
    \item We will use two arrays to keep track of global and local
      index a kernel function can see from a work item respectively.
    \item We declare two arrays as {\tt int globalId[2][N][N]} and
      {\tt int localId[2][N][N]}.
  \end{itemize}
\end{frame}

\begin{frame}
  \frametitle{Kernel}
  \programlisting{get-global-local-id.cl}{}{\scriptsize}
\end{frame}

\begin{frame}
  \frametitle{Kernel}
  \begin{itemize}
  \item We first call {\tt get\_global\_id(0)} and {\tt
    get\_global\_id(1)} to know the indices of this work item in {\tt
    NDRange}, and put them into {\tt id0} and {\tt id1}.
  \item Then we call {\tt get\_global\_id} and {\tt get\_local\_id} to
    get the indices.
  \end{itemize}
\end{frame}

\begin{frame}
  \frametitle{Buffers}
  \begin{itemize}
    \item In the main program we declare two host buffers {\tt
      bufferGlobalId} and {\tt bufferLocalId} to hold the global and
      local indices.
    \item This buffer will link to parameters {\tt globalId[2][N][N]}
      and {\tt localId[2][N][N]} in the kernel function.
  \end{itemize}
\end{frame}

\begin{frame}
  \frametitle{Buffers}
  \programlistingfirst{get-global-local-id.c}{}{\scriptsize}{createbuffer}{setshape}
\end{frame}

\begin{frame}
  \frametitle{Dimension}
  \begin{itemize}
    \item Now we set the dimension of {\tt NDRange} to 2, and set the
      size of each dimension to {\tt N}, as in {\tt globalDim}.
    \item We set the size of dimension of a work group from the
      command line arguments, and place the sizes into {\tt localDim}.
    \item We place the kernel into the command queue, then call {\tt
      clFinish} to wait for the completion.
  \end{itemize}
\end{frame}

\begin{frame}
  \frametitle{Buffers}
  \programlistingfirst{get-global-local-id.c}{}{\scriptsize}{setshape}{check}
\end{frame}

\begin{frame}
  \frametitle{Demonstration}
  \begin{itemize}
    \item Run the {\tt get-global-id-cl} program by setting the work
      group size to be 4 by 4.
    \item Run the {\tt get-global-id-cl} program by setting the work
      group size to be 2 by 4.
  \end{itemize}
\end{frame}

\begin{frame}
  \frametitle{Discussion}
  \begin{itemize}
  \item Describe the printed indices under different group sizes.
  \end{itemize}
\end{frame}

\subsection{Matrix Multiplication}

\begin{frame}
  \frametitle{Matrix Multiplication} 
  \begin{itemize}
    \item We now use matrix multiplication as an example of using
      global index.
    \item We will multiply two $N$ by $N$ matrices using GPU.
  \end{itemize}
\end{frame}

\begin{frame}
  \frametitle{Kernel}
  \begin{itemize}
    \item The multiplication kernel has three parameters -- {\tt A},
      {\tt B} and {\tt C}.
    \item We will multiply {\tt A} with {\tt B}, and place the results
      in {\tt C}.
    \item We first get the row and column number of this work item,
      then preform an inner product.
  \end{itemize}
\end{frame}

\begin{frame}
  \frametitle{Kernel}
  \programlisting{mul-kernel.cl}{}{\scriptsize}
\end{frame}

\begin{frame}
  \frametitle{Matrices}
  \begin{itemize}
    \item The main program first prepares host memory matrices {\tt A}
      and {\tt B}.
  \end{itemize}
\end{frame}

\begin{frame}
  \frametitle{Buffers}
  \programlistingfirst{matrixMul.c}{}{\scriptsize}{vector}{createbuffer}
\end{frame}

\begin{frame}
  \frametitle{Buffers}
  \begin{itemize}
    \item The main program then creates OpenCL buffers for {\tt A},
      {\tt B} and {\tt C}.
    \item The contents of A and B will be copied into devices
      ({\tt CL\_MEM\_COPY\_HOST\_PTR}), and the GPU will use host memory buffer
      directly ({\tt CL\_MEM\_USE\_HOST\_PTR}).
  \end{itemize}
\end{frame}

\begin{frame}
  \frametitle{Buffers}
  \programlistingfirst{matrixMul.c}{}{\scriptsize}{createbuffer}{setarg}
\end{frame}

\begin{frame}
  \frametitle{Run Kernel}
  \begin{itemize}
  \item We set the size of both dimensions to {\tt N}, as in {\tt
    globalDim}.
  \item We set the sizes of dimensions of a work group as 1 by 1 and
    place the sizes into {\tt localDim}.
  \item We place the kernel into the command queue, then call {\tt
    clFinish} to wait for the completion.
  \end{itemize}
\end{frame}

\begin{frame}
  \frametitle{Buffers}
  \programlistingfirst{matrixMul.c}{}{\scriptsize}{setshape}{checkandfree}
\end{frame}



\begin{frame}
  \frametitle{Demonstration}
  \begin{itemize}
    \item Run the {\tt matrixMul-cl} program.
  \end{itemize}
\end{frame}

\begin{frame}
  \frametitle{Discussion}
  \begin{itemize}
  \item Did you notice the significant speed difference in the
    computation and verification parts?
  \end{itemize}
\end{frame}

\section{Time Measurement}

\subsection{Event}

\begin{frame}
  \frametitle{Time Measurement}
  \begin{itemize}
    \item We would like to measure the kernel execution time.
    \item The kernel is sent to a device through a {\em command
      queue}, so we need to enable the command queue for profiling.
    \item We associate an event with the end of a kernel execution,
      then wait for the event.
    \item We retrieve the timing information from the event.
  \end{itemize}
\end{frame}

\begin{frame}
  \frametitle{Steps}
  \begin{itemize}
    \item We set the property of the command queue to allow profiling.
    \item We then declare a variable of type {\tt cl\_event} for
      the event.
    \item We then supply the event as the last parameter when calling
      {\tt clEnqueueNDRangeKernel}.
    \item We then wait for this event by calling a function {\tt
      clWaitForEvents}.
    \item Finally we extract the timing information from the event by
      calling {\tt clGetEventProfilingInfo}.
  \end{itemize}
\end{frame}

\begin{frame}
  \frametitle{Command Queue}
  \begin{itemize}
    \item The kernel is sent to a device through a {\em command
      queue}.
    \item If we want to time (profile) the kernel execution, we need
      to explicitly set the property of the command queue to {\tt
        CL\_PROFILING\_COMMAND\_QUEUED}.  
  \end{itemize}
\end{frame}

\begin{frame}
  \prototypedetail{clCreateCommandQueue.h}{\scriptsize}
\end{frame}

\begin{frame}
  \frametitle{Property}
  \begin{itemize}
    \item We explicitly set the {\tt CL\_PROFILING\_COMMAND\_QUEUED}
      property of the command queue when we created it.
    \item Note that the API needs a list of property/value pairs, which must end in 0.
    \item This {\em list} is actually an array.
  \end{itemize}
\end{frame}

\begin{frame}
  \programlistingfirst{matrixMul-time.c}{}{\footnotesize}{commandqueue}{kernelsource}
\end{frame}

\begin{frame}
  \frametitle{Discussion}
  \begin{itemize}
    \item Google to find out other properties that can be set for a
      command queue.
  \end{itemize}
\end{frame}

\begin{frame}
  \frametitle{Kernel Execution}
  \begin{itemize}
  \item After setting the command queue for profiling, we can start
    the kernel.
  \item We supply {\tt event} as the last parameter while calling
    {\tt clEnqueueNDRangeKernel}, so now we have associated the
    event with the kernel execution.
  \end{itemize}
\end{frame}

\begin{frame}
  \prototypedetail{clEnqueueNDRangeKernel.h}{\scriptsize}
\end{frame}

\begin{frame}
  \programlistingfirst{matrixMul-time.c}{}{\footnotesize}{setshape}{waitforevent}
\end{frame}

\begin{frame}
  \frametitle{Wait for Event}
  \begin{itemize}
    \item Now we have submitted the kernel to execution, we just wait
      for the event to happen.
    \item We call {\tt clWaitForEvents} to wait for event(s).
  \end{itemize}
\end{frame}

\begin{frame}
  \frametitle{clWaitForEvents}
  \prototypedetail{clWaitForEvents.h}{\footnotesize}
\end{frame}

\begin{frame}
  \frametitle{Parameters}
  \begin{description}
  \item [\tt num\_events] The number of events to wait for.
  \item [\tt event\_list] The array of events to wait for.
  \end{description}
\end{frame}

\begin{frame}
  \programlistingfirst{matrixMul-time.c}{}{\footnotesize}{waitforevent}{gettime}
\end{frame}

\begin{frame}
  \frametitle{Discussion}
  \begin{itemize}
    \item Trace the {\tt matrixMul-time.c} to understand the flow so far.
  \end{itemize}
\end{frame}

\begin{frame}
  \frametitle{Three Stages}
  A kernel execution will go through three stages.
  \begin{itemize}
    \item First the kernel joined the command queue for execution.
    \item When it is the turn of the kernel, it is submitted to the
      device for execution.
    \item When the kernel finished execution, it will trigger the
      event you associated with it.
  \end{itemize}
\end{frame}

\begin{frame}
  \frametitle{Get Information}
  The event can provide the following four times.
  \begin{itemize}
    \item The time when the kernel joined the command queue.
    \item The time when the kernel was sent to the device for execution.
    \item The time when the kernel started execution.
    \item The time when the kernel finished execution.
  \end{itemize}
\end{frame}

\begin{frame}
  \frametitle{Get Information}
  \begin{itemize}
    \item We call {\tt clGetEventProfilingInfo} to know the four
      times.
    \item The time-stamp is of type {\tt cl\_ulong}.
  \end{itemize}
\end{frame}

\begin{frame}
  \frametitle{\tt clGetEventProfilingInfo}
  \prototypedetail{clGetEventProfilingInfo.h}{\footnotesize}
\end{frame}

\begin{frame}
  \frametitle{Parameters}
  \begin{description}
  \item [\tt event] The event that we want to query.
  \item [\tt param\_name] The information to query.
  \item [\tt param\_value\_size] The length of buffer ({\tt
    param\_value}) to store the answer.
  \item [\tt param\_value] The buffer to store the answer.
  \item [\tt param\_value\_size\_ret] The actual length of the
    returned answer.
  \end{description}
\end{frame}

\begin{frame}
  \frametitle{Four Time Points}
  \begin{itemize}
  \item We put the four times in four variables -- {\tt
    timeEnterQueue}, {\tt timeSubmit}, {\tt timeStart}, and {\tt
    timeEnd}.
  \end{itemize}
\end{frame}

\begin{frame}
  \frametitle{Time Calculation}
  \programlistingfirst{matrixMul-time.c}{}{\scriptsize}{gettime}{printtime}
\end{frame}

\begin{frame}
  \frametitle{Time Calculation}
  \begin{itemize}
  \item We calculate the duration of each stage by the difference of
    the beginning and the ending time-stamps.
  \item The unit of time is nano ($10^{-9}$) second. 
  \end{itemize}
\end{frame}

\begin{frame}
  \programlistingfirst{matrixMul-time.c}{}{\footnotesize}{printtime}{checkandfree}
\end{frame}

\begin{frame}
  \frametitle{Demonstration}
  \begin{itemize}
    \item Run the {\tt matrixMul-time-cl} program.
  \end{itemize}
\end{frame}

\begin{frame}
  \frametitle{Discussion}
  \begin{itemize}
  \item Observe the times of the three stages.
  \end{itemize}
\end{frame}

\subsection{Example}

\begin{frame}
  \frametitle{Buffer Comparison}
  \begin{itemize}
    \item Now we know how to measure kernel execution time, we can
      compare the effects of different communication buffers on the
      execution time.
    \item We will compare the executive time of using {\tt
      CL\_MEM\_COPY\_HOST\_PTR} and {\tt CL\_MEM\_USE\_HOST\_PTR} for
      creating {\tt A} and {\tt B} buffers.
    \item We will fix the allocation method of {\tt C} to make a
      meaningful comparison.
  \end{itemize}
\end{frame}

\begin{frame}
  \frametitle{Difference}
  \begin{itemize}
    \item The only difference between the following two programs is
      how they allocate {\tt A} and {\tt B} matrices.
  \end{itemize}
\end{frame}

\begin{frame}
  \programlistingfirst{matrixMul-time-copy.c}{}{\scriptsize}{createbuffer}{setarg}
\end{frame}

\begin{frame}
  \programlistingfirst{matrixMul-time-use.c}{}{\scriptsize}{createbuffer}{setarg}
\end{frame}

\begin{frame}
  \frametitle{Demonstration}
  \begin{itemize}
    \item Run the {\tt matrixMul-time-copy-cl} and {\tt
      matrixMul-time-use-cl} programs.
  \end{itemize}
\end{frame}

\begin{frame}
  \frametitle{Discussion}
  \begin{itemize}
  \item Compare the execution times of the two programs.
  \item What is the reason for this difference in kernel execution
    time?
  \end{itemize}
\end{frame}

\section{Local Memory}

\begin{frame}
  \frametitle{Local Memory}
  \begin{itemize}
    \item Local memory is shared by processing units in the same work
      group.
    \item Local memory is fast but small.
    \item We will use local memory to speed up matrix multiplication.
  \end{itemize}
\end{frame}

\begin{frame}
  \frametitle{Idea}
  \begin{itemize}
    \item Both matrices $A$ and $B$ are $N$ by $N$.
    \item We will partition the matrices into $Blk$ by $Blk$ blocks,
      so each block has $N / Blk$ rows and columns.
    \item For ease of notation we will use $Block(A, i, j)$ to denote
      the block in $i$-th row of blocks and $j$-th column of blocks in
      $A$.
  \end{itemize}
\end{frame}

\begin{frame}
  \frametitle{Work Group}
  \begin{itemize}
  \item Each work group will compute a block in $C$.
  \item Each thread will compute an element of $C$.
  \end{itemize}
\end{frame}

\begin{frame}
  \frametitle{Divide and Conquer}
  \begin{itemize}
  \item Now consider the work group that is responsible for
    computing the $Block(C, i, j)$.
  \item This work group will first multiply $Block(A, i, 1)$ by
    $Block(B, 1, j)$.
  \item This work group will then multiply $Block(A, i, 2)$ by
    $Block(B, 2, j)$.
  \item ...
  \item This work group will then multiply $Block(A, i, N)$ by
    $Block(B, N, j)$.
  \item Then $Block(C, i, j)$ is the sum of all these products.
  \end{itemize}
\end{frame}

\begin{frame}
  \frametitle{Divide and Conquer}
  \centerline{\pgfimage[height=0.6\textheight]{matrix-mul.pdf}}
\end{frame}

\begin{frame}
  \frametitle{Discussion}
  \begin{itemize}
  \item Make sure that you understand this algorithm.
  \end{itemize}
\end{frame}

\begin{frame}
  \frametitle{Fast multiplication}
  \begin{itemize}
    \item If we know how to do matrix multiplication on two blocks, we
      know how to solve the whole problem.
    \item The problem is that if we do this on global memory, it will
      be slow.
    \item The idea is to bring $Block(A, i, 1)$ and $Block(B, 1,
      j)$ into local memory, then we can multiply them fast.
    \item How??
  \end{itemize}
\end{frame}

\begin{frame}
  \frametitle{Memory Movement}
  \begin{itemize}
  \item There are $(N / Blk)^2$ elements in $Block(A, i, 1)$.
  \item There are $(N / Blk)^2$ threads in this work group.
  \item We make each thread to move an element from both $Block(A, i,
    1)$ $Block(B, 1, j)$ into local memory, then from both $Block(A,
    i, 2)$ and $Block(B, 2, j)$, and so on.
  \item After each movement each thread computes an element in
    $Block(C, i, j)$ using the data in local memory.
  \end{itemize}
\end{frame}

\begin{frame}
  \frametitle{Move to Local Memory}
  \centerline{\pgfimage[height=0.8\textheight]{matrix-mul-move.pdf}}
\end{frame}

\begin{frame}
  \frametitle{Profitable}
  \begin{itemize}
    \item Why is this profitable?
    \item Each thread only moves two data.
    \item Each thread will do a vector inner product on two vector of
      length $(N / Blk)$.
    \item That is, every data moved into local memory is shared by
      $(N / Blk)$ other threads.
  \end{itemize}
\end{frame}

\begin{frame}
  \frametitle{Discussion}
  \begin{itemize}
    \item Make sure that you understand this profitable theory.
  \end{itemize}
\end{frame}

\begin{frame}
  \frametitle{Steps} Now from a thread, or a kernel point of view, it
  will go through the following steps.
  \begin{itemize}
    \item Get the global and local indices of this work item.
    \item Go through $Blk$ iterations, where each of these
      iterations multiplies two blocks.
      \begin{itemize}
        \item Move a data from A in global memory into local memory.
        \item Move a data from B in global memory into local memory.
        \item After both steps are done compute the inner product and
          add it to a variable {\tt sum}
      \end{itemize}
    \item Put sum into the corresponding $C$ element.
  \end{itemize}
\end{frame}

\begin{frame}
  \frametitle{Constants}
  \begin{itemize}
    \item Since the kernel function file cannot include other source
      files, we can only define these constants here. A more
      consistent method should be used in the future.
    \item We define a symbol {\tt BSIDE} for the size of a side of the
      a block.
  \end{itemize}
\end{frame}

\begin{frame}
  \programlistingfirst{mul-local-kernel.cl}{}{}{constant}{mul}
\end{frame}

\begin{frame}
  \frametitle{Interface}
  \begin{itemize}
    \item We use global memory as the interface between the kernel and
      the host, and the interface is the same as before.
    \item We declare two local matrices in local memory, using the
      {\tt \_\_local} keyword.
  \end{itemize}
\end{frame}

\begin{frame}
  \programlistingfirst{mul-local-kernel.cl}{}{}{mul}{loop}
\end{frame}

\begin{frame}
  \frametitle{Data Movement}
  \begin{itemize}
    \item Each thread (kernel function) moves one data into local $A$
      and $B$.
    \item However, we need to make sure that when we start the inner
      product, {\em all} the data are there.
    \item Remember a thread only moves two data, other data are moved
      by other threads -- we do not know if they have finished or not.
    \item We need to synchronize with all other threads in the work group.
  \end{itemize}
\end{frame}

\begin{frame}
  \programlistingfirst{mul-local-kernel.cl}{}{\footnotesize}{loop}{inner}
\end{frame}

\begin{frame}
  \frametitle{Move to Local Memory}
  \centerline{\pgfimage[height=0.8\textheight]{matrix-mul-move.pdf}}
\end{frame}


\begin{frame}
  \frametitle{Discussion}
  \begin{itemize}
    \item Do we need to synchronize with threads in other work groups?
    \item Convince yourself that the index calculation is correct.
  \end{itemize}
\end{frame}

\begin{frame}
  \frametitle{Synchronization}
  \begin{itemize}
    \item We use {\tt barrier} to synchronize threads.
    \item All threads in the same group will synchronize.
  \end{itemize}
\end{frame}

\begin{frame}
  \frametitle{barrier}
  \prototypedetail{barrier.h}{\small}
\end{frame}

\begin{frame}
  \frametitle{Parameters}
  \begin{description}
  \item [\tt flags] The memory level this synchronization guarantees.
    \begin{description}
      \item [\tt CLK\_LOCAL\_MEM\_FENCE] Guarantees that all local
        memory operations will finish.
      \item [\tt CLK\_GLOBAL\_MEM\_FENCE] Guarantees that all global
        memory operations will finish.
    \end{description}
  \end{description}
\end{frame}

\begin{frame}
  \programlistingfirst{mul-local-kernel.cl}{}{\footnotesize}{loop}{end}
\end{frame}

\begin{frame}
  \frametitle{C}
  \begin{itemize}
    \item The answer in $C$ is accumulated throughout the iterations in
      variable {\tt sum}.
    \item We place another barrier after the inner product.
    \item We only update local memory so we use {\tt CLK\_LOCAL\_MEM\_FENCE}.
  \end{itemize}
\end{frame}

\begin{frame}
  \frametitle{C}
  \centerline{\pgfimage[height=0.6\textheight]{matrix-mul-inner.pdf}}
\end{frame}

\begin{frame}
  \frametitle{Demonstration}
  \begin{itemize}
    \item Run the {\tt matrixMul-time-copy-local-cl} program.
  \end{itemize}
\end{frame}

\begin{frame}
  \frametitle{Discussion}
  \begin{itemize}
    \item Do we need both synchronizations?
    \item Observe the timing.
  \end{itemize}
\end{frame}

\end{document}
