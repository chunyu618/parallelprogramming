\documentclass{beamer}
\usetheme{Warsaw}

\usepackage{color}
\usepackage{listings}
\usepackage{url}
\usepackage{pgf}


% global setting

\newcommand{\comment}[1]{}

% list environment gloabal setting

% to specify the size of fonts in lst environment

\lstset{showstringspaces=false}
\lstset{language=C}
\lstset{basicstyle=\tt}
\lstset{keywordstyle=\color{blue}}
\lstset{emphstyle=\color{red}}
\lstset{numbers=left}
\lstset{frame=single}
\lstset{rangeprefix=\/*\ }
\lstset{rangesuffix=\ *\/}
\lstset{includerangemarker=false}

%% slides related macros

\newenvironment{mytitle}[2]
{\begin{frame}{#1} \vspace{0.2\textheight}{\huge #2} \end{frame}}

%% Macro for non-numbered frames listing 
%% \nonumberlisting{title}{file}{options}

\newcommand{\nonumberlisting}[3]
{\noindent{#1}
{\lstinputlisting[#3]{#2}}}

%% Program listing macros

%% \programlisting{file}{desp}{font}
%% \programlistingfirst{file}{desp}{font}{start}{end}
%% \programlistingrest{file}{descp}{font}{start}{end}


%% List a program in one page with a font size.

\newcommand{\programlisting}[3]
{\renewcommand{\lstlistingname}{#3 \bf Example}
\lstinputlisting[basicstyle={#3 \tt},caption={#3 \bf (#1) #2}]{#1}}

%% List a program for the first page within a given range.

\newcommand{\programlistingfirst}[5]
{\renewcommand{\lstlistingname}{#3 \bf Example}
\lstinputlisting[basicstyle={#3 \tt},linerange={#4-#5},caption={#3 \bf (#1) #2}]{#1}}

%% List a program for the second page (or later).

\newcommand{\programlistingrest}[5]
{\lstinputlisting[basicstyle={#3 \tt},linerange={#4-#5}]{#1}}

%% List a program in two pages.

\newcommand{\programlistingtwoslides}[6]
{\begin{frame}                  % do not add [fragile]
\programlistingfirst{#1}{#2}{#3}{#4}{#5}
\end{frame}
\begin{frame}
\programlistingrest{#1}{#2}{#3}{#5}{#6}
\end{frame}}

%% List a program in three pages.

\newcommand{\programlistingthreeslides}[7]
{\begin{frame}                  % do not add [fragile]
\programlistingfirst{#1}{#2}{#3}{#4}{#5}
\end{frame}
\begin{frame}
\programlistingrest{#1}{#2}{#3}{#5}{#6}
\end{frame}
\begin{frame}
\programlistingrest{#1}{#2}{#3}{#6}{#7}
\end{frame}
}


%% Program input/output macros

%% \programoutput{file}
%% \programoutputfirst{file}{font}{options}
%% \programoutputrest{file}{font}{startline}{endline}
%% \programoutputtwoslides{file}{font}{start}{endl}{start2}{end2}

%% Program output macor with default setting

\newcommand{\programoutput}[1]
{\programoutputfirst{#1}{}{}}

%% Program output macor for the first page

\newcommand{\programoutputfirst}[3]
{\lstset{basicstyle={#2 \tt}}
\nonumberlisting{#2 \bf 輸出}{#1.out}{#3}}

%% Program output macor for later page

\newcommand{\programoutputrest}[4]
{\lstinputlisting[firstnumber={#3},basicstyle={\tt #2},linerange={#3-#4}]
{#1.out}}

%% Program output macor for two slides

\newcommand{\programoutputtwoslides}[6]
{\begin{frame}                  % do not add [fragile]
\programoutputfirst{#1}{#2}{linerange={#3-#4}}
\end{frame}
\begin{frame}
\programoutputrest{#1}{#2}{#5}{#6}
\end{frame}}


%% \programinput{file}
%% \programinputfirst{file}{font}{option}

%% Program input macor with default setting

\newcommand{\programinput}[1]
{\programinputfirst{#1}{}{}}

%% Program input macor for the first page

\newcommand{\programinputfirst}[3]
{\lstset{basicstyle={#2 \tt}}
\nonumberlisting{#2 \bf 輸入}{#1.in}{#3}}


%% \piotwocol{file}

\newcommand{\piotwocol}[1]{\piotwocoldetail{#1}{}{}}

\newcommand{\piotwocoldetail}[2]{
\begin{columns}[t,onlytextwidth]
\begin{column}{0.45\textwidth}
\programinputfirst{#1}{#2}{}
\end{column}
\begin{column}{0.45\textwidth}
\programoutputfirst{#1}{#2}{}
\end{column}
\end{columns}}

%% \piotworow{file}

\newcommand{\piotworow}[1]
{\programinput{#1}
\programoutput{#1}}

\newcommand{\piotworowdetail}[2]
{\programinputfirst{#1}{#2}{}
\programoutputfirst{#1}{#2}{}}


%% \newcommand{\headerfile}[1]
%% {\nonumberlisting {{\bf 標頭檔} {\tt #1}}{#1}{}}

\newcommand{\header}[2]{\headerdetail{#1}{#2}{}}

\newcommand{\headerdetail}[3]
{{\renewcommand{\lstlistingname}{#3 \bf Header}
\lstset{basicstyle={#3 \tt}}
\lstset{label=header:#1,caption={(#1) #2}}
\lstinputlisting{#1}}}

%% Macro for prototype
%% \prototype{file}

\newcommand{\prototype}[1]{\prototypedetail{#1}{}}

%% Macro for prototype with detialed setting
%% \prototypedetail{file}{font}

\newcommand{\prototypedetail}[2]
{{\renewcommand{\lstlistingname}{#2 \bf Prototype}
\lstset{basicstyle={#2 \tt}}
\lstset{frame=single,label=prototype:#1,caption={#2}}
\lstinputlisting{#1}}}

%% Macro for phrase
%% \nphrase{file}{desc}

%% \newcommand{\nphrase}[2]
%% {{\renewcommand{\lstlistingname}{\lstfontsize \bf 片語}
%% \lstset{label=phr:#1,frame=single,caption={\lstfontsize #2}}
%% \lstset{basicstyle={\lstfontsize \tt}}
%% \lstinputlisting{#1.p}}
%% }

\newcommand{\nphrase}[2]
{\nphrasedetail{#1}{#2}{}}

%% Macro for detail phrase
%% \nphrasedetail{file}{desc}{font}

\newcommand{\nphrasedetail}[3]
{{\renewcommand{\lstlistingname}{#3 \bf 片語}
\lstset{label=phr:#1,frame=single,caption={#3 #2}}
\lstset{basicstyle={#3 \tt}}
\lstinputlisting{#1.p}}}



\newcommand{\inputfile}[1]
{\nonumberlisting {{\bf 輸入檔} {\tt #1}}{#1}{}}

\newcommand{\outputfile}[1]
{\nonumberlisting {{\bf 輸出檔} {\tt #1}}{#1}{}}

% Macro for listings that are labeled.
% [1] is the filename, [2] is the description

\newcommand{\commandline}[1]
{\lstset{basicstyle={\tt}}
\nonumberlisting{\bf 命令列}{#1.bat}{}}

% {\boxedlisting{\bf 命令列}{#1.bat}}

% Macro for listings that are labeled.
% [1] is the filename, [2] is the description

\newcommand{\out}[2]
{{\renewcommand{\lstlistingname}{\bf 執行結果}
\lstset{label=out:#1,caption=#2}
\lstinputlisting{#1.out}}}






\begin{document}

\title{Basic OpenCL Programming}

\author{Pangfeng Liu \\ National Taiwan University}

\begin{frame}
  \titlepage
\end{frame}

\section{Introduction}

\begin{frame}
  \frametitle{OpenCL} 
  \begin{itemize}
  \item Open Computing Language (OpenCL) is a framework for writing
    programs that execute across heterogeneous platforms consisting of
    central processing units (CPUs), graphics processing units (GPUs),
    digital signal processors (DSPs), field-programmable gate arrays
    (FPGAs) and other processors.\footnote{\url
      http://en.wikipedia.org/wiki/OpenCL}
  \end{itemize}
\end{frame}

\begin{frame}
  \frametitle{Model} 
  \begin{itemize}
  \item OpenCL specifies a language (based on C99) for programming these
    devices and application programming interfaces (APIs) to control the
    platform and execute programs on the compute devices. 
  \item OpenCL provides parallel computing using task-based and
    data-based parallelism.
  \end{itemize}
\end{frame}

\begin{frame}
  \frametitle{Standard}
  \begin{itemize}
  \item OpenCL is an open standard maintained by the non-profit
    technology consortium Khronos Group.  
  \item Conformant implementations are available from Altera, AMD,
    Apple, ARM Holdings, Creative Technology, IBM, Imagination
    Technologies, Intel, Nvidia, Qualcomm, Samsung, Vivante, Xilinx, and
    ZiiLABS.
  \end{itemize}
\end{frame}

\begin{frame}
  \frametitle{OpenCL Vendors}
  \centerline{\pgfimage[width=0.9\textwidth]{SPIR5.png}}\footnote{\url{http://images.anandtech.com/doci/7161/SPIR5.png}}
\end{frame}

\begin{frame}
  \frametitle{Discussion}
  \begin{itemize}
  \item Describe the relation between parallel programming and OpenCL.
  \item Google ``SPIR'' and describe its importance and its
    relation to OpenCL.
  \item What will happen if the implementation of an ``integer'' is
    different among the devices?
  \end{itemize}
\end{frame}

\subsection{GPGPU}

\begin{frame}
  \frametitle{GPGPU}
  \begin{itemize}
  \item General-purpose computing on graphics processing units (GPGPU)
    is the use of a graphics processing unit (GPU), which typically
    handles computation only for computer graphics, to perform
    computation in applications traditionally handled by the central
    processing unit
    (CPU).\footnote{\url{http://en.wikipedia.org/wiki/General-purpose_computing_on_graphics_processing_units}}
  \end{itemize}
\end{frame}

\begin{frame}
  \begin{itemize}
  \item The use of multiple graphics cards in one computer, or large
    numbers of graphics chips, further parallelizes the already parallel
    nature of graphics processing.
  \item In addition, even a single GPU-CPU framework provides advantages
    that multiple CPUs on their own do not offer due to specialization
    in each chip.
  \end{itemize}
\end{frame}

\begin{frame}
  \frametitle{Advantages}
  \begin{itemize}
  \item Fast vector processing for data parallel algorithms on regular
    data structures.
  \item High performance-to-price ratio for cost effective high
    performance computing.
  \item Leverage the fast growing 3D graphics rendering technology for
    high performance computing.
  \end{itemize}
\end{frame}

\begin{frame}
  \frametitle{Discussion}
  \begin{itemize}
  \item Compare the price, clock rate, and performance (measured in
    floating point operations per second) of GPU and CPU.
  \item What is the reason that GPU usually deal with ``regular
    computation''?  Give your reasoning.
  \end{itemize}
\end{frame}

\begin{frame}
  \frametitle{History}
  \begin{itemize}
  \item General-purpose computing on GPUs only became practical and
    popular after ca. 2001, with the advent of both programmable
    shaders and floating point support on graphics processors.
  \item In particular, problems involving matrices and/or vectors were
    easy to translate to a GPU, which acts with native speed and
    support on those types.
  \end{itemize}
\end{frame}

\begin{frame}
  \frametitle{History}
  \begin{itemize}
  \item The scientific computing community's experiments with the new
    hardware started with a matrix multiplication routine (2001); one
    of the first common scientific programs to run faster on GPUs than
    CPUs was an implementation of LU factorization (2005).
  \end{itemize}
\end{frame}

\begin{frame}
  \frametitle{Pretend to be Pixels}
  \begin{itemize}
  \item The early efforts to use GPUs as general-purpose processors
    required reformulating computational problems in terms of graphics
    primitives, as supported by the two major APIs for graphics
    processors, OpenGL and DirectX. 
  \item This cumbersome translation was obviated by the advent of
    general-purpose programming languages and APIs such as Sh/RapidMind,
    Brook and Accelerator.
  \end{itemize}
\end{frame}

\begin{frame}
  \frametitle{No Need to Pretend}
  \begin{itemize}
  \item Nvidia's CUDA allows programmers to ignore the underlying
    graphical concepts in favor of more common high-performance
    computing concepts.
  \item Newer, hardware vendor-independent offerings include Microsoft's
    DirectCompute and Apple/Khronos Group's OpenCL.
  \item Modern GPGPU pipelines can act on any "big data" operation and
    leverage the speed of a GPU without requiring full and explicit
    conversion of the data to a graphical form.
  \end{itemize}
\end{frame}

\begin{frame}
  \frametitle{Discussion}
  \begin{itemize}
  \item Find out the full name of CUDA.
  \item What does the name imply?
  \end{itemize}
\end{frame}

\subsection{Heterogeneity}

\begin{frame}
  \frametitle{Heterogeneity}
  \begin{itemize}
  \item Modern computer systems consist of different types of
    processing units, e.g., CPU and GPU.
  \item Different processing units have different characteristics.
    \begin{itemize}
    \item CPU is suitable for complex control, e.g., recursion,
      complex branching, etc.
    \item GPU is suitable for regular computation patterns, e.g.,
      linear algebra and matrix manipulation.
    \end{itemize}
  \end{itemize}
\end{frame}

\begin{frame}
  \frametitle{Advantages}
  \begin{itemize}
  \item Different processing units for different applications according
    to their requirements.
  \item Flexibility for dispatching jobs.
  \item Energy conservation while maintaining quality of service
    simultaneously.
  \end{itemize}
\end{frame}

\begin{frame}
  \frametitle{Management Issues}
  \begin{itemize}
  \item Different processing units has different ISA's and requires
    different binary codes.
  \item A single programming model may not be flexible enough to
    cover all the processing units.
  \item Serious compiler support for heterogeneity among processing
    units.
  \end{itemize}
\end{frame}

\begin{frame}
  \frametitle{Discussion}
  \begin{itemize}
  \item Why does a heterogeneous computing environment is more
    complicated than a homogeneous one?  Give your reasoning.
  \end{itemize}
\end{frame}

\subsection{HSA in the Future}

\begin{frame}
  \frametitle{Heterogeneous System Architecture}
  \begin{itemize}
  \item Heterogeneous System Architecture (HSA) is a computer
    processor architecture that integrates central processing units
    and graphics processors on the same bus, with shared memory and
    tasks.\footnote{\url{http://en.wikipedia.org/wiki/Heterogeneous_System_Architecture}}
  \item Heterogeneous computing itself refers to systems that contain
    multiples processing units – central processing units (CPUs),
    graphics processing units (GPUs), digital signal processors
    (DSPs), or any type of application-specific integrated circuits
    (ASICs).
  \end{itemize}
\end{frame}

\begin{frame}
  \frametitle{Unified View}
  \begin{itemize}
  \item The HSA is being developed by the HSA Foundation, which includes
    (among many others) AMD and ARM. 
  \item HSA is to reduce communication latency between CPUs, GPUs and
    other compute devices, and make these various devices more
    compatible from a programmer's perspective, relieving the programmer
    of the task of planning the moving of data between devices' disjoint
    memories, which is done by OpenCL or CUDA right now.
  \end{itemize}
\end{frame}

\begin{frame}
  \frametitle{Features}
  \begin{itemize}
  \item HSA defines a unified virtual address space space for
    compute devices: where GPUs traditionally have their own memory,
    separate from the main (CPU) memory.
  \item HSA requires these devices to share page tables so that
    devices can exchange data by sharing pointers, which is to be
    supported by custom memory management units.  
  \item To render interoperability possible and also to ease
    various aspects of programming, HSA is intended to be
    ISA-agnostic for both CPUs and accelerators, and to support
    high-level programming languages.
  \end{itemize}
\end{frame}

\begin{frame}
  \frametitle{Discussion}
  \begin{itemize}
  \item What do you think are the key issues in the success of HSA?
    Give your reasoning.
  \end{itemize}
\end{frame}

\section{OpenCL Models}

\subsection{Hardware}

\begin{frame}
  \frametitle{Hierarchy}
  \begin{itemize}
  \item An OpenCL computing system consisting of a {\em host} and {\em
    computing resources}.
  \item The computing resources consist of a number of {\em compute
    devices}\footnote{might be central processing units (CPUs) or
    ``accelerators'' such as graphics processing units (GPUs).}
  \item A single compute device typically consists of many {\em compute
    units}.
  \item A single compute unit typically consists of {\em processing
    elements} (PEs).
  \end{itemize}
\end{frame}

\begin{frame}
  \frametitle{OpenCL Hardware Model}
  \centerline{\pgfimage[width=0.9\textwidth]{opencl_platform_model.png}}\footnote{\url{http://www.rastergrid.com/blog/wp-content/uploads/2010/11/opencl_platform_model.png}}
\end{frame}

\begin{frame}
  \frametitle{Host}
  \begin{itemize}
  \item The host machine is a CPU.
  \item The computation is dispatched to computing resources for
    execution.
  \end{itemize}
\end{frame}

\begin{frame}
  \frametitle{Computational Resources}
  \begin{enumerate}
  \item Compute device 1
    \begin{enumerate}
    \item Compute unit 1
      \begin{enumerate}
      \item Processing unit 1
      \item Processing unit 2
      \item ...
      \end{enumerate}
    \item Compute unit 2
    \item ...
    \end{enumerate}
  \item Compute device 2
  \item ...
  \end{enumerate}
\end{frame}

\begin{frame}
  \frametitle{Host v.s. Device}
  \begin{itemize}
  \item Host
    \begin{itemize}
    \item Sequential code written in C/C++
    \item For coordination
    \item Single threaded
    \end{itemize}
  \item Device
    \begin{itemize}
    \item Parallel code written in OpenCL 
    \item For computation
    \item Multi-threaded
    \end{itemize}
  \end{itemize}
\end{frame}

\begin{frame}
  \frametitle{Discussion}
  \begin{itemize}
  \item Describe the three layer architecture in OpenCL hardware
    architecture.
  \end{itemize}
\end{frame}

\subsection{Data}

\begin{frame}
  \frametitle{Domain and Kernel}
  \begin{itemize}
  \item The computation is defined on a N-dimensional {\em domain}
    called {\tt NDRange}.
  \item You can think of domain as a N-dimensional array, and each
    array element stores a data.
  \item We perform {\em kernel} computation on these array elements in
    a data parallel fashion.
  \end{itemize}
\end{frame}

\begin{frame}
  \frametitle{Hierarchy}
  \begin{itemize}
  \item The entire computation domain is divided into {\em work groups}.
  \item A work group is divided into {\em work items}.
  \item A kernel describes the work to be done for a work item.
  \end{itemize}
\end{frame}

\begin{frame}
  \frametitle{Hierarchy}
  \centerline{\pgfimage[width=0.9\textwidth]{Basic_Concepts.jpg}}\footnote{\url{https://software.intel.com/sites/landingpage/opencl/optimization-guide/OG_files/Basic_Concepts.jpg}}
\end{frame}


\begin{frame}
  \frametitle{Mapping}
  \begin{itemize}
  \item The entire computation domain is run on the OpenCL devices.
  \item A work group is run on a compute unit.
  \item A work item is run on a compute element.
  \end{itemize}
\end{frame}

\begin{frame}
  \frametitle{Discussion}
  \begin{itemize}
  \item Describe the three layer architecture in OpenCL data domain.
  \end{itemize}
\end{frame}

\subsection{Memory}

\begin{frame}
  \frametitle{Memory}
  \begin{itemize}
  \item The OpenCL memory is consistent with the hardware hierarchy.
  \item A processing unit has its own private memory.
  \item Processing units in the same compute unit share a local memory.
  \item Processing units in the same compute device share a global memory.
  \end{itemize}
\end{frame}

\begin{frame}
  \frametitle{Memory}
  \centerline{\pgfimage[width=0.9\textwidth]{memory.png}}\footnote{\url{http://1.bp.blogspot.com/-KLUr3r0tOt8/UoIWBwwqSGI/AAAAAAAABzA/WWXaDG4KAkg/s1600/img10.png}}
\end{frame}

\begin{frame}
  \frametitle{Host v.s. Device}
  \begin{itemize}
  \item Different memory system from the host.
  \item Host cannot address the memory on device and device cannot
    address the memory on host, so memory copying is necessary.
  \end{itemize}
\end{frame}

\begin{frame}
  \frametitle{Performance}
  \begin{itemize}
  \item The global memory is larger than the local memory, which is
    larger than the private memory.
  \item The private memory is faster than the local memory, which is
    faster than the global memory.
  \end{itemize}
\end{frame}

\begin{frame}
  \frametitle{Memory Movement}
  \begin{itemize}
  \item Data must be given to the host by I/O operation.
  \item Data is then transfer from the host to the device global memory
    for processing.
  \item Optionally one can move the data from global into local
    memory for faster access.
  \item Even more optionally one can move the data from the local
    memory into Private memory for even faster processing.
  \end{itemize}
\end{frame}

\begin{frame}
  \frametitle{Memory Movement}
  \begin{itemize}
  \item OK now data processing finishes, you need to move data {\em
    all the way back} to the host for visualization.
  \item This is tedious -- hopefully later architecture like HSA
    will give us a unified, and better, view on memory, from both
    GPU and CPU.
  \end{itemize}
\end{frame}

\begin{frame}
  \frametitle{Mapping}
  \begin{center}
    \begin{tabular}{|l|l|l|} \hline
      hardware & programming & memory  \\  \hline \hline
      host & host data & host memory   \\  \hline \hline
      compute device & computation domain & global memory  \\  \hline
      compute unit & work group & local memory \\ \hline 
      processing unit & work item & private memory \\ \hline
    \end{tabular}
  \end{center}
\end{frame}

\begin{frame}
  \frametitle{Discussion}
  \begin{itemize}
  \item Describe the three layer architecture in OpenCL memory view.
  \end{itemize}
\end{frame}

\section{Information Query}

\subsection{Platform}

\begin{frame}
  \frametitle{Platform and Device}
  \begin{itemize}
  \item We start with querying the system to understand its
    configuration.
  \item The first thing we do is to know the number of platforms the
    system has.
  \item A system can have many platforms, and each platform can
    have many devices.
  \end{itemize}
\end{frame}

\begin{frame}
  \prototypedetail{clGetPlatformIDs.h}{\footnotesize}
\end{frame}

\begin{frame}
  \frametitle{clGetPlatformIDs}
  \begin{description}
  \item [\tt num\_entries] The size of platform ids provided by the
    next parameter {\tt platforms}, where the platform ids will be
    added into.  It must be positive if {\tt platforms} is not {\tt
      NULL}.
  \item [\tt platforms] An array of {\tt cl\_platform\_id} found. 
  \item [\tt num\_platforms] The number of platforms available in the
    system.  {\tt NUll} will be ignored.
  \end{description}
\end{frame}

\begin{frame}
  \frametitle{\tt clGetPlatformIDs}
  \programlisting{getPlatformID.c}{}{\footnotesize}
\end{frame}

\begin{frame}
  \frametitle{Compilation}
  \begin{itemize}
  \item Our platform is a NVIDIA implementation of OpenCL.
  \item In order to compile OpenCL we need to specify OpenCL headers
    and library.
  \end{itemize}
\end{frame}

\begin{frame}
  \frametitle{Makefile}
  \programlisting{Makefile}{}{\scriptsize}
\end{frame}

\begin{frame}
  \frametitle{Demonstration}
  \begin{itemize}
  \item Run the {\tt getPlatformID-cl} program.
  \end{itemize}
\end{frame}

\begin{frame}
  \frametitle{Discussion}
  \begin{itemize}
  \item How many platform do we have in the system?
  \end{itemize}
\end{frame}

\begin{frame}
  \frametitle{Platform Information}
  \begin{itemize}
    \item After knowing the number of platform and their ids, we want
      to know their information.
    \item OpenCL provides a function {\tt clGetPlatformInfo} for this.
  \end{itemize}
\end{frame}

\begin{frame}
  \prototypedetail{clGetPlatformInfo.h}{\footnotesize}
\end{frame}

\begin{frame}
  \frametitle{Parameters}
  \begin{description}
  \item [\tt platform] The id of the platform you want to query.
  \item [\tt param\_name] The property you want to query.
  \item [\tt param\_value\_size] The length of {\tt param\_value} in
    bytes.
  \item [\tt param\_value] The location for the query answer.
  \item [\tt param\_value\_size\_ret] The number of bytes returned
    from the query.
  \end{description}
\end{frame}

\begin{frame}
  \frametitle{Request}
  We may request the following information from a platform.
  \begin{description}
  \item[\tt CL\_PLATFORM\_PROFILE]
  \item[\tt CL\_PLATFORM\_VERSION]
  \item[\tt CL\_PLATFORM\_NAME]
  \item[\tt CL\_PLATFORM\_VENDOR]
  \item[\tt CL\_PLATFORM\_EXTENSIONS]
  \end{description}
\end{frame}

\begin{frame}
  \frametitle{Platform ID}
  \begin{itemize}
  \item We first call {\tt clGetPlatformIDs} to get the ids into array
    {\tt platform\_id}.
  \item The actual number of platforms found will be in {\tt
    platform\_id\_got}.
  \item We then use the platform id to find out its properties.
  \end{itemize}
\end{frame}

\programlistingtwoslides{getPlatformInfo.c}{}{\scriptsize}{header}{getinfo}{end}

\begin{frame}
  \frametitle{Platform Information}
  \begin{itemize}
  \item The actual number of platforms found is in {\tt
    platform\_id\_got}.
  \item The platform ids are in {\tt platform\_id}.
  \item We loop through all platform ids and use {\tt
    clGetPlatformInfo} to retrieve the property we need.
  \item The id will be in the {\tt buffer} we provided, and the length
    of query result will be in {\tt length}.
  \item We are not sure about anything after the specified length so
    we place a zero character at the end.  This may not be necessary.
  \end{itemize}
\end{frame}

\begin{frame}
  \frametitle{Platform Property}
  \begin{itemize}
  \item We request the name, the vendor, the OpenCL version, and the
    profile of the platform.
  \item A full profile means a full implementation of OpenCL.
  \end{itemize}
\end{frame}


\begin{frame}
  \frametitle{Demonstration}
  \begin{itemize}
  \item Run the {\tt getPlatformInfo-cl} program.
  \end{itemize}
\end{frame}

\begin{frame}
  \frametitle{Discussion}
  \begin{itemize}
  \item Does the implementation place an zero character at the end of
    the answer?
  \item Google {\tt CL\_PLATFORM\_EXTENSIONS} and find out its
    meaning.
  \end{itemize}
\end{frame}

\subsection{Device}

\begin{frame}
  \frametitle{Device ID and Information}
  \begin{itemize}
    \item After knowing platform ids we can find out the ids of the
      devices in that platform.
    \item After knowning the device id we can know the information
      about a device.
  \end{itemize}
\end{frame}

\begin{frame}
  \prototypedetail{clGetDeviceIDs.h}{\footnotesize}
\end{frame}

\begin{frame}
  \begin{description}
  \item [\tt platform] The id of the platform you want to query.
  \item [\tt device\_type] The type of device you want to query.
  \item [\tt num\_entries] The number of devices that can be added into
    the buffer provided by the next parameter {\tt devices}.
  \item [\tt devices] A pointer to the buffer to store the devices.
  \item [\tt num\_devices] The actual number of devices returned.
  \end{description}
\end{frame}

\begin{frame}
  \frametitle{Device Types}
  There are five types of devices one can query.
  \begin{description}
  \item[CL\_DEVICE\_TYPE\_CPU]
  \item[CL\_DEVICE\_TYPE\_GPU]
  \item[CL\_DEVICE\_TYPE\_ACCELERATOR]
  \item[CL\_DEVICE\_TYPE\_DEFAULT]
  \item[CL\_DEVICE\_TYPE\_ALL]
  \end{description}
\end{frame}

\programlistingthreeslides{getDeviceID.c}{}{\scriptsize}{header}{getinfo}{getDeviceID}{end}

\begin{frame}
  \frametitle{Device IDs}
  \begin{itemize}
  \item The number of device id returned is in {\tt device\_id\_got},
    and the ids are in {\tt devices}.
  \item We test all device types, i.e. CPU type, and GPU type.
  \end{itemize}
\end{frame}

\begin{frame}
  \frametitle{Demonstration}
  \begin{itemize}
  \item Run the {\tt getDeviceID-cl} program.
  \end{itemize}
\end{frame}

\begin{frame}
  \frametitle{Discussion}
  \begin{itemize}
  \item Does the platform have anything other than CPU and GPU?
  \end{itemize}
\end{frame}

\begin{frame}
  \frametitle{clGetDeviceInfo}
  \begin{itemize}
  \item After receiving the device ids, we can query the detailed
    information about a device.
  \item The procedure is very similar to {\tt clGetPlatformInfo}.
  \end{itemize}
\end{frame}

\begin{frame}
  \prototypedetail{clGetDeviceInfo.h}{\footnotesize}
\end{frame}

\begin{frame}
  \frametitle{clGetDeviceInfo}
  \begin{description}
  \item [\tt device] The id of the device you want to query.
  \item [\tt param\_name] The property you want to query.
  \item [\tt param\_value\_size] The size of the answer that will be
    returned.
  \item [\tt param\_value] The buffer for the answer.
  \item [\tt param\_value\_size\_ret] The actual size of answer returned.
  \end{description}
\end{frame}

\begin{frame}
  \programlistingfirst{getDeviceInfo.c}{}{\scriptsize}{getDeviceInfo}{end}
\end{frame}

\begin{frame}
  \frametitle{Device Information}
  \begin{itemize}
  \item We know the number of devices from the previous {\tt
    device\_id\_got}.
  \item We first query the name of the device, which is given as a
    character buffer.
  \item The we query the global memory size, local memory size, the
    maximum number of compute units, and the maximum number of
    work groups.
  \item These numbers are of type {\tt cl\_ulong}, which is defined by
    OpenCL.
  \end{itemize}
\end{frame}

\begin{frame}
  \frametitle{Demonstration}
  \begin{itemize}
  \item Run the {\tt getDeviceInfo-cl} program.
  \end{itemize}
\end{frame}

\begin{frame}
  \frametitle{Discussion}
  \begin{itemize}
  \item Google what can be queried from {\tt clGetDeviceInfo}.  Make a
    list of property that you think is useful.
  \end{itemize}
\end{frame}

\section{Program Execution}

\subsection{Context}

\begin{frame}
  \frametitle{Vector Add}
  \begin{itemize}
  \item We will use a simple example to work through the execution
    of OpenCL program.
  \item This example will add two vectors with GPU.
  \item We first query the system to get the platform and device
    information we need.
  \end{itemize}
\end{frame}

\begin{frame}
  \frametitle{Platform and Device}
  \programlistingfirst{vectorAdd.c}{}{}{header}{main}
\end{frame}

\begin{frame}
  \frametitle{Headers}
  \begin{itemize}
  \item We need to include OpenCL header file.
  \item The length of the vector is {\tt N}.
  \item The maximum length of the kernel source is {\tt MAXK}
  \end{itemize}
\end{frame}

\begin{frame}
  \frametitle{Platform and Device}
  \programlistingfirst{vectorAdd.c}{}{\scriptsize}{main}{getcontext}
\end{frame}

\begin{frame}
  \frametitle{Platform and Device}
  \begin{itemize}
  \item We then get only one device, and all of its GPUs.
  \item Note that we need to check the return status.
  \end{itemize}
\end{frame}

\begin{frame}
  \frametitle{Platform and Device}
  \centerline{\pgfimage[height=0.9\textheight]{deviceID.pdf}}
\end{frame}

\begin{frame}
  \frametitle{Discussion}
  \begin{itemize}
  \item Read the code and identify the part that we will use only GPU.
  \end{itemize}
\end{frame}

\begin{frame}
  \frametitle{clCreateContext}
  \begin{itemize}
  \item After knowing the platform and device information, we can build
    a {\em context} to run our OpenCL applications.
  \item The context consists mostly the devices that will participate
    this computation.
  \end{itemize}
\end{frame}

\begin{frame}
  \prototypedetail{clCreateContext.h}{\scriptsize}
\end{frame}

\begin{frame}
  \frametitle{Parameters}
  \begin{description}
  \item [\tt properties] A list of properties for this context.
  \item [\tt num\_devices] The number of devices this context will
    use. The devices are in the next parameter {\tt devices}.
  \item [\tt devices] The devices used by this context.
  \item [\tt pfn\_notify] A callback routine for error.
  \item [\tt user\_data] A parameter that will be supplied to the
    callback routine {\tt pfn\_notify}.
  \item [\tt errcode\_ret] The error code.
  \end{description}
\end{frame}

\begin{frame}
  \frametitle{Simplicity}
  \begin{itemize}
  \item For simplicity in this lecture we will not use callback
    function and property.
  \end{itemize}
\end{frame}

\begin{frame}
  \frametitle{Context}
  \programlistingfirst{vectorAdd.c}{}{\scriptsize}{getcontext}{commandqueue}
\end{frame}

\begin{frame}
  \frametitle{Context}
  \centerline{\pgfimage[height=0.9\textheight]{context.pdf}}
\end{frame}

\begin{frame}
  \frametitle{Discussion}
  \begin{itemize}
  \item Read the code and identify the part that the context we
    build uses all GPUs.
  \end{itemize}
\end{frame}

\subsection{CommandQueue}

\begin{frame}
  \frametitle{Command Queue}
  \begin{itemize}
  \item Within a context the kernel will start execution and send
    commands to devices.
  \item We need to build a command queue from the host to a device so
    that the command from the kernel can be sent there.
  \item A command queue connects to a device.
  \end{itemize}
\end{frame}

\begin{frame}
  \prototypedetail{clCreateCommandQueue.h}{\small}
\end{frame}

\begin{frame}
  \frametitle{Parameters}
  \begin{description}
  \item [\tt context] The context this command queue belongs to.
  \item [\tt device] The device connected by this command queue. It has
    to be one of the devices in the context.
  \item [\tt properties] A pointer to the properties.
  \item [\tt errcode\_ret] The error code.
  \end{description}
\end{frame}

\begin{frame}
  \frametitle{Simplicity}
  \begin{itemize}
  \item For simplicity in this lecture we will not set property.
  \item Note that the property is a pointer so we can set it to NULL.
  \end{itemize}
\end{frame}

\begin{frame}
  \frametitle{Command Queue}
  \programlistingfirst{vectorAdd.c}{}{\scriptsize}{commandqueue}{kernelsource}
\end{frame}

\begin{frame}
  \frametitle{Command Queue}
  \centerline{\pgfimage[height=0.9\textheight]{command_queue.pdf}}
\end{frame}

\begin{frame}
  \frametitle{Discussion}
  \begin{itemize}
  \item Which GPU does this command queue connect to?
  \end{itemize}
\end{frame}

\subsection{Program}

\begin{frame}
  \frametitle{Kernel Program}
  \begin{itemize}
  \item We must specify the computation as a kernel.
  \item The source of a kernel is in the form of string, not file.
  \item OpenCL provides functions to build kernel program from strings.
  \item For ease of kernel modification we will place the kernel
    source into a file ``kernel.cl''.
  \end{itemize}
\end{frame}

\begin{frame}
  \prototypedetail{clCreateProgramWithSource.h}{\footnotesize}
\end{frame}

\begin{frame}
  \frametitle{Parameters}
  \begin{description}
  \item [\tt context] The context for which to build the program.
  \item [\tt count] The number of strings in the program.
  \item [\tt strings] A pointer array to store the strings.  That is,
    your kernel program may consist of many strings.
  \item [\tt lengths] An array of string lengths for the pointer array
    {\tt strings}.
  \item [\tt errcode\_ret] The error code.
  \end{description}
\end{frame}

\begin{frame}
  \frametitle{Kernel Source}
  \programlistingfirst{vectorAdd.c}{}{\scriptsize}{kernelsource}{buildprogram}
\end{frame}

\begin{frame}
  \frametitle{From File to String}
  \begin{itemize}
  \item We store our kernel in a file ``kernel.cl''.
  \item We read the contents of ``kernel.cl'' into a buffer {\tt
    kernelBuffer} with fread.  We also know the length of the file
    by the return value of fread.
  \item The we call {\tt clCreateProgramWithSource} with the {\tt
    kernelBuffer}.  Note that we only have one string.
  \end{itemize}
\end{frame}

\begin{frame}
  \frametitle{Program}
  \centerline{\pgfimage[height=0.9\textheight]{program.pdf}}
\end{frame}

\begin{frame}
  \frametitle{Discussion}
  \begin{itemize}
  \item Identify the part of the code that determines the length of
    the kernel string.
  \end{itemize}
\end{frame}

\begin{frame}
  \frametitle{Compile and Build}
  \begin{itemize}
  \item After building your program object from a string, you need to
    compile and link it.
  \item OpenCL provides a {\tt clBuildProgram} to do this.
  \end{itemize}
\end{frame}

\begin{frame}
  \prototypedetail{clBuildProgram.h}{\footnotesize}
\end{frame}

\begin{frame}
  \frametitle{Parameters}
  \begin{description}
  \item [\tt program] The program to be build.
  \item [\tt num\_devices] The number of devices to be used in {\tt
    device\_list}.
  \item [\tt device\_list] The device list.
  \item [\tt options] The build option to be used.
  \item [\tt pfn\_notify] The callback function for error.
  \item [\tt user\_data] The parameter supplied to {\tt pfn\_notify}.
  \end{description}
\end{frame}


\begin{frame}
  \frametitle{Build Program}
  \programlistingfirst{vectorAdd.c}{}{\scriptsize}{buildprogram}{createkernel}
\end{frame}

\begin{frame}
  \frametitle{Discussion}
  \begin{itemize}
  \item Identify the GPUs for which we have compiled and built the
    program.
  \end{itemize}
\end{frame}

\subsection{Kernel}

\begin{frame}
  \frametitle{Kernel}
  \begin{itemize}
  \item After building the program object, we are ready to build
    kernel.
  \item We only need the program object to do this.
  \end{itemize}
\end{frame}

\begin{frame}
  \prototypedetail{clCreateKernel.h}{\footnotesize}
\end{frame}

\begin{frame}
  \frametitle{Parameters}
  \begin{description}
  \item [\tt program] The kernel program we built.
  \item [\tt kernel\_name] The name of the kernel function.
  \item [\tt errcode\_ret] Error code.
  \end{description}
\end{frame}

\begin{frame}
  \frametitle{Create Kernel}
  \programlistingfirst{vectorAdd.c}{}{\scriptsize}{createkernel}{vector}
\end{frame}

\begin{frame}
  \frametitle{Kernel}
  \centerline{\pgfimage[height=0.9\textheight]{kernel.pdf}}
\end{frame}

\begin{frame}
  \frametitle{Discussion}
  \begin{itemize}
  \item Look into file kernel.cl and make sure that the kernel
    function name there matches the name we provided while creating
    the kernel.
  \end{itemize}
\end{frame}

\subsection{Buffer}

\begin{frame}
  \frametitle{Vectors}
  \begin{itemize}
  \item We now prepare the input vectors {\tt A} and {\tt B}, and
    the output vector {\tt C} in host memory.
  \item We place them on heap and properly initialize them.
  \end{itemize}
\end{frame}

\begin{frame}
  \frametitle{Vector}
  \programlistingfirst{vectorAdd.c}{}{\scriptsize}{vector}{createbuffer}
\end{frame}

\begin{frame}
  \frametitle{Discussion}
  \begin{itemize}
    \item Observe the code and determine the correct value of {\tt C}.
  \end{itemize}
\end{frame}

\begin{frame}
  \frametitle{Buffer}
  \begin{itemize}
  \item Devices cannot access the memory of the host directly.
  \item After the host receives the input from I/O, it needs to
    create a buffer object and link it with the host memory the data
    is in, so the device can access the data.  
  \item OpenCL provide this linkage by a pointer to {\tt cl\_mem},
    i.e., a buffer.
  \end{itemize}
\end{frame}

\begin{frame}
  \prototypedetail{clCreateBuffer.h}{\small}
\end{frame}

\begin{frame}
  \frametitle{Parameters}
  \begin{description}
  \item [\tt context] The context this buffer belongs to.
  \item [\tt flags] Properties for this buffer.
  \item [\tt size] The size of the buffer.
  \item [\tt host\_ptr] The memory on the host this buffer refers to.
  \item [\tt errcode\_ret] Error code.
  \end{description}
\end{frame}

\begin{frame}
  \frametitle{Buffer Property}
  We can set the following property for a buffer.
  \begin{description}
  \item[CL\_MEM\_READ\_ONLY] The buffer is read only.
  \item[CL\_MEM\_WRITE\_ONLY] The buffer is write only.
  \item[CL\_MEM\_READ\_WRITE] We can read and write the buffer
  \item[CL\_MEM\_USE\_HOST\_PTR] OpenCL uses the host memory as the buffer.
  \item[CL\_MEM\_COPY\_HOST\_PTR] OpenCL copies the contents of the
    host memory into a buffer accessible from GPU.
  \end{description}
\end{frame}

\begin{frame}
  \frametitle{Create Buffer}
  \begin{itemize}
  \item We set the buffer for {\tt A} and {\tt B} to be read only,
    and the buffer for {\tt C} to be write only.
  \item Note that we need to check the status for errors.
  \end{itemize}
\end{frame}

\begin{frame}
  \frametitle{Vector}
  \programlistingfirst{vectorAdd.c}{}{\scriptsize}{createbuffer}{setarg}
\end{frame}


\begin{frame}
  \frametitle{Buffer}
  \centerline{\pgfimage[height=0.9\textheight]{buffer.pdf}}
\end{frame}

\begin{frame}
  \frametitle{Discussion}
  \begin{itemize}
    \item Observe the flag we give to each buffer.  Why some of them
      are read only and some of them are write only?
  \end{itemize}
\end{frame}


\begin{frame}
  \frametitle{Parameter Linking}
  \begin{itemize}
  \item Now we want to connect the parameters kernel will see with the
    buffers we provide.
  \item This is set by the order the parameters appearing in the
    prototype.  That is, we need to make sure the order is A, B, then
    C.  Please refer to the following kernel source.
  \end{itemize}
\end{frame}

\begin{frame}
  \programlisting{kernel.cl}{}{\footnotesize}
\end{frame}

\begin{frame}
  \prototypedetail{clSetKernelArg.h}{\footnotesize}
\end{frame}

\begin{frame}
  \frametitle{Parameters}
  \begin{description}
  \item [\tt kernel] The kernel to set the argument.
  \item [\tt arg\_index] the index of the argument (staring from 0).
  \item [\tt arg\_size] The location of the argument.
  \item [\tt arg\_value] The size of the argument.
  \end{description}
\end{frame}

\begin{frame}
  \frametitle{Mapping}
  \programlistingfirst{vectorAdd.c}{}{\scriptsize}{setarg}{setshape}
\end{frame}

\begin{frame}
  \frametitle{Discussion}
  \begin{itemize}
    \item Observe the code and make sure that the order of parameters
      is correct.
  \end{itemize}
\end{frame}

\begin{frame}
  \frametitle{Shape of Data}
  \begin{itemize}
  \item Now we need to define the shape of the data domain we will
    work on.
  \item First we determine the dimension of the data domain.
  \item Then we determine the size of each dimension of the global
    domain, and the size of each dimension of the a work group.
  \item These two information also determine the number of work groups.
  \end{itemize}
\end{frame}

\begin{frame}
  \prototypedetail{clEnqueueNDRangeKernel.h}{\scriptsize}
\end{frame}

\begin{frame}
  \frametitle{Parameters}
  \begin{description}
  \item [\tt command\_queue] The commandqueue to send the kernel code.
  \item [\tt kernel] the kernel to run.
  \item [\tt work\_dim] The dimension of work items.
  \item [\tt global\_work\_offset] {\tt NULL} for now.
  \item [\tt global\_work\_size] An array of {\tt work\_dim} integers
    that specify the dimension of the global data domain.
  \end{description}
\end{frame}

\begin{frame}
  \frametitle{Parameters}
  \begin{description}
  \item [\tt local\_work\_size] An array of {\tt work\_dim} integers
    that specify dimensions of a work group.
  \item [\tt num\_events\_in\_wait\_list] The number of events this
    computation must wait for completion.
  \item [\tt event\_wait\_list] The list of events to wait for.
  \item [\tt event] Returns an event the describes the result of this
    computation.
  \end{description}
\end{frame}

\begin{frame}
  \frametitle{Data Shape}
  \programlistingfirst{vectorAdd.c}{}{\scriptsize}{setshape}{getcvector}
\end{frame}

\begin{frame}
  \frametitle{Domain Shape}
  \begin{itemize}
  \item We set the domain as one dimensional.  The size is {\tt N}.
  \item The size of a work group is set to 1, so there will be $N$
    work groups.  Each work group has one work item.
  \end{itemize}
\end{frame}

\begin{frame}
  \frametitle{\tt NDRange}
  \centerline{\pgfimage[height=0.9\textheight]{NDRange.pdf}}
\end{frame}

\begin{frame}
  \frametitle{Discussion}
  \begin{itemize}
    \item How many work groups do we have?
    \item How many work items do we have?
  \end{itemize}
\end{frame}

\begin{frame}
  \frametitle{Get Result}
  \begin{itemize}
  \item Finally we need to get the result from buffer back to host
    array {\tt C}
  \item We accomplish this by placing a read buffer command into the
    command queue.  
  \item When the GPU runs this command the buffer will be copied
    into {\tt C}.
  \end{itemize}
\end{frame}

\begin{frame}
  \prototypedetail{clEnqueueReadBuffer.h}{\footnotesize}
\end{frame}

\begin{frame}
  \frametitle{Parameters}
  \begin{description}
  \item [\tt command\_queue] The command queue to place the read
    buffer command.
  \item [\tt buffer] The buffer to read.
  \item [\tt blocking\_read] If the read is blocking.
  \item [\tt offset] The offset in buffer to read.
  \item [\tt cb] The amount of data to read.
  \item [\tt ptr] The location of host memory the data will be read
    into.
  \end{description}
\end{frame}

\begin{frame}
  \frametitle{Parameters}
  \begin{description}
  \item [\tt num\_events\_in\_wait\_list] The number of events this
    computation must wait for completion.
  \item [\tt event\_wait\_list] The list of events to wait for.
  \item [\tt event] Returns an event the decibels the result of this
    computation.
  \end{description}
\end{frame}

\begin{frame}
  \frametitle{Get Results}
  \programlistingfirst{vectorAdd.c}{}{\footnotesize}{getcvector}{checkandfree}
\end{frame}

\begin{frame}
  \frametitle{Read Buffer}
  \centerline{\pgfimage[height=0.9\textheight]{readbuffer.pdf}}
\end{frame}


\begin{frame}
  \frametitle{Receive Results}
  \begin{itemize}
  \item We set the mode to be blocking because we can only check the
    result when all data are ready.
  \item We read the data from the {\em beginning} so we set the offset to 0.
  \end{itemize}
\end{frame}

\begin{frame}
  \frametitle{Discussion}
  \begin{itemize}
    \item Which GPU does this command queue connect to?
  \end{itemize}
\end{frame}

\begin{frame}
  \frametitle{Check}
  \begin{itemize}
  \item Finally the host checks the correctness of the computation.
  \item The host also releases the objects it created.
    \begin{itemize}
      \item The buffers {\tt A}, {\tt B}, and {\tt C}.
      \item Other objects related to OpenCL.
        \begin{itemize}
        \item context
        \item program
        \item kernel
        \item command queue
        \item buffers
        \end{itemize}
    \end{itemize}
  \end{itemize}
\end{frame}

\begin{frame}
  \frametitle{Release Object} The object releasing API is as
  follow. The only parameter is the object you want to release.
  \begin{itemize}
  \item {\tt clReleaseContext}
  \item {\tt clReleaseCommandQueue}
  \item {\tt clReleaseProgram}
  \item {\tt clReleaseKernel}
  \item {\tt clReleaseMemObject}
  \end{itemize}
\end{frame}

\begin{frame}
  \frametitle{Get Results}
  \programlistingfirst{vectorAdd.c}{}{\scriptsize}{checkandfree}{end}
\end{frame}

\begin{frame}
  \frametitle{Demonstration}
  \begin{itemize}
  \item Run the {\tt vectorAdd-cl} program.
  \end{itemize}
\end{frame}

\begin{frame}
  \frametitle{Discussion}
  \begin{itemize}
    \item Have we released all objects we created?
  \end{itemize}
\end{frame}

\end{document}
