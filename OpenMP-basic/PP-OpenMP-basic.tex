\documentclass{beamer}
\usetheme{Warsaw}

\usepackage{color}
\usepackage{listings}
\usepackage{url}
\usepackage{pgf}


% global setting

\newcommand{\comment}[1]{}

% list environment gloabal setting

% to specify the size of fonts in lst environment

\lstset{showstringspaces=false}
\lstset{language=C}
\lstset{basicstyle=\tt}
\lstset{keywordstyle=\color{blue}}
\lstset{emphstyle=\color{red}}
\lstset{numbers=left}
\lstset{frame=single}
\lstset{rangeprefix=\/*\ }
\lstset{rangesuffix=\ *\/}
\lstset{includerangemarker=false}

%% slides related macros

\newenvironment{mytitle}[2]
{\begin{frame}{#1} \vspace{0.2\textheight}{\huge #2} \end{frame}}

%% Macro for non-numbered frames listing 
%% \nonumberlisting{title}{file}{options}

\newcommand{\nonumberlisting}[3]
{\noindent{#1}
{\lstinputlisting[#3]{#2}}}

%% Program listing macros

%% \programlisting{file}{desp}{font}
%% \programlistingfirst{file}{desp}{font}{start}{end}
%% \programlistingrest{file}{descp}{font}{start}{end}


%% List a program in one page with a font size.

\newcommand{\programlisting}[3]
{\renewcommand{\lstlistingname}{#3 \bf Example}
\lstinputlisting[basicstyle={#3 \tt},caption={#3 \bf (#1) #2}]{#1}}

%% List a program for the first page within a given range.

\newcommand{\programlistingfirst}[5]
{\renewcommand{\lstlistingname}{#3 \bf Example}
\lstinputlisting[basicstyle={#3 \tt},linerange={#4-#5},caption={#3 \bf (#1) #2}]{#1}}

%% List a program for the second page (or later).

\newcommand{\programlistingrest}[5]
{\lstinputlisting[basicstyle={#3 \tt},linerange={#4-#5}]{#1}}

%% List a program in two pages.

\newcommand{\programlistingtwoslides}[6]
{\begin{frame}                  % do not add [fragile]
\programlistingfirst{#1}{#2}{#3}{#4}{#5}
\end{frame}
\begin{frame}
\programlistingrest{#1}{#2}{#3}{#5}{#6}
\end{frame}}

%% List a program in three pages.

\newcommand{\programlistingthreeslides}[7]
{\begin{frame}                  % do not add [fragile]
\programlistingfirst{#1}{#2}{#3}{#4}{#5}
\end{frame}
\begin{frame}
\programlistingrest{#1}{#2}{#3}{#5}{#6}
\end{frame}
\begin{frame}
\programlistingrest{#1}{#2}{#3}{#6}{#7}
\end{frame}
}


%% Program input/output macros

%% \programoutput{file}
%% \programoutputfirst{file}{font}{options}
%% \programoutputrest{file}{font}{startline}{endline}
%% \programoutputtwoslides{file}{font}{start}{endl}{start2}{end2}

%% Program output macor with default setting

\newcommand{\programoutput}[1]
{\programoutputfirst{#1}{}{}}

%% Program output macor for the first page

\newcommand{\programoutputfirst}[3]
{\lstset{basicstyle={#2 \tt}}
\nonumberlisting{#2 \bf 輸出}{#1.out}{#3}}

%% Program output macor for later page

\newcommand{\programoutputrest}[4]
{\lstinputlisting[firstnumber={#3},basicstyle={\tt #2},linerange={#3-#4}]
{#1.out}}

%% Program output macor for two slides

\newcommand{\programoutputtwoslides}[6]
{\begin{frame}                  % do not add [fragile]
\programoutputfirst{#1}{#2}{linerange={#3-#4}}
\end{frame}
\begin{frame}
\programoutputrest{#1}{#2}{#5}{#6}
\end{frame}}


%% \programinput{file}
%% \programinputfirst{file}{font}{option}

%% Program input macor with default setting

\newcommand{\programinput}[1]
{\programinputfirst{#1}{}{}}

%% Program input macor for the first page

\newcommand{\programinputfirst}[3]
{\lstset{basicstyle={#2 \tt}}
\nonumberlisting{#2 \bf 輸入}{#1.in}{#3}}


%% \piotwocol{file}

\newcommand{\piotwocol}[1]{\piotwocoldetail{#1}{}{}}

\newcommand{\piotwocoldetail}[2]{
\begin{columns}[t,onlytextwidth]
\begin{column}{0.45\textwidth}
\programinputfirst{#1}{#2}{}
\end{column}
\begin{column}{0.45\textwidth}
\programoutputfirst{#1}{#2}{}
\end{column}
\end{columns}}

%% \piotworow{file}

\newcommand{\piotworow}[1]
{\programinput{#1}
\programoutput{#1}}

\newcommand{\piotworowdetail}[2]
{\programinputfirst{#1}{#2}{}
\programoutputfirst{#1}{#2}{}}


%% \newcommand{\headerfile}[1]
%% {\nonumberlisting {{\bf 標頭檔} {\tt #1}}{#1}{}}

\newcommand{\header}[2]{\headerdetail{#1}{#2}{}}

\newcommand{\headerdetail}[3]
{{\renewcommand{\lstlistingname}{#3 \bf Header}
\lstset{basicstyle={#3 \tt}}
\lstset{label=header:#1,caption={(#1) #2}}
\lstinputlisting{#1}}}

%% Macro for prototype
%% \prototype{file}

\newcommand{\prototype}[1]{\prototypedetail{#1}{}}

%% Macro for prototype with detialed setting
%% \prototypedetail{file}{font}

\newcommand{\prototypedetail}[2]
{{\renewcommand{\lstlistingname}{#2 \bf Prototype}
\lstset{basicstyle={#2 \tt}}
\lstset{frame=single,label=prototype:#1,caption={#2}}
\lstinputlisting{#1}}}

%% Macro for phrase
%% \nphrase{file}{desc}

%% \newcommand{\nphrase}[2]
%% {{\renewcommand{\lstlistingname}{\lstfontsize \bf 片語}
%% \lstset{label=phr:#1,frame=single,caption={\lstfontsize #2}}
%% \lstset{basicstyle={\lstfontsize \tt}}
%% \lstinputlisting{#1.p}}
%% }

\newcommand{\nphrase}[2]
{\nphrasedetail{#1}{#2}{}}

%% Macro for detail phrase
%% \nphrasedetail{file}{desc}{font}

\newcommand{\nphrasedetail}[3]
{{\renewcommand{\lstlistingname}{#3 \bf 片語}
\lstset{label=phr:#1,frame=single,caption={#3 #2}}
\lstset{basicstyle={#3 \tt}}
\lstinputlisting{#1.p}}}



\newcommand{\inputfile}[1]
{\nonumberlisting {{\bf 輸入檔} {\tt #1}}{#1}{}}

\newcommand{\outputfile}[1]
{\nonumberlisting {{\bf 輸出檔} {\tt #1}}{#1}{}}

% Macro for listings that are labeled.
% [1] is the filename, [2] is the description

\newcommand{\commandline}[1]
{\lstset{basicstyle={\tt}}
\nonumberlisting{\bf 命令列}{#1.bat}{}}

% {\boxedlisting{\bf 命令列}{#1.bat}}

% Macro for listings that are labeled.
% [1] is the filename, [2] is the description

\newcommand{\out}[2]
{{\renewcommand{\lstlistingname}{\bf 執行結果}
\lstset{label=out:#1,caption=#2}
\lstinputlisting{#1.out}}}






\begin{document}

\title{Basic OpenMP}

\author{Pangfeng Liu \\ National Taiwan University}

\begin{frame}
\titlepage
\end{frame}

\section{Introduction} 

\begin{frame}
\frametitle{OpenMP}
\begin{itemize}
\item OpenMP~\footnote{\url{http://openmp.org/}} (Open
  Multiprocessing) is an API that supports multi-platform shared
  memory multiprocessing programming in C, C++, and Fortran, on most
  processor architectures and operating systems, including Solaris,
  AIX, HP-UX, GNU/Linux, Mac OS X, and Windows platforms.
\end{itemize}
\end{frame}

\begin{frame}
\frametitle{OpenMP} 
\begin{itemize}
\item Implemented as compiler preprocessor directive.
\begin{itemize}
\item All OpenMP constructs can be safely ignored without affecting
  the semantic of the original program.
\end{itemize}
\item Easy to understand and easy to use.
\begin{itemize}
\item OpenMP is at a much high level than Pthread and handles all
  parallelization details.
\end{itemize}
\item Promote ``incremental'' parallelism.
\begin{itemize}
\item One can start with a working sequential program and ``upgrade''
  it to a parallel program with minimum efforts.
\end{itemize}
\end{itemize}
\end{frame}

\begin{frame}
\frametitle{Hello, world}
\begin{itemize}
\item Just like the case in Pthread, we start with a sequential
  program that prints hello world.
\item Unlike the case in Pthread, we will only print it once.
\end{itemize}
\end{frame}

% hello

\begin{frame}
\frametitle{Hello World} 
\programlisting{hello}{Hello, world!}
\end{frame}

\section{OpenMP Threads}

\subsection{\tt \#pragma omp parallel}

\begin{frame}
\frametitle{\tt \#pragma omp} 
\begin{itemize}
\item {\tt \#pragma omp} is a preprocessor directive.
\item All OpenMP directives start with {\tt \#pragma omp}.
\item The effect of a OpenMP directive only applies to the {\em next}
  statement.
\item This directive instructs the compiler to generate OpenMP
  parallel code if it supports OpenMP.
\item This directive will be ignored by the compiler if OpenMP is not
  supported.
\end{itemize}
\end{frame}

\begin{frame}
\frametitle{\tt \#pragma omp parallel} 
\begin{itemize}
\item When we use gcc compilation option {\tt -fopenmp} to compile the
  program, {\tt \#pragma omp parallel} will instructs the compiler to
  generate code to run the statement on {\em every} core.
\begin{itemize}
\item For example, if the CPU has 4 cores, then {\tt \#pragma omp
  parallel} will spawn 4 threads, and each of them runs {\tt printf}
  on a core.
\end{itemize}
\item If we do not use {\tt -fopenmp} then gcc will just generate a
  sequential code.
\end{itemize}
\end{frame}

\begin{frame}
\frametitle{Hello World} 
\filelisting{compile.sh}{compilation}
\end{frame}

\begin{frame}
\frametitle{OpenMP} 
\begin{itemize}
\item The compiler can generate both the sequential and parallel
  code depending on the compilation flag.
\item Incremental parallelism by having a working sequential code
  first, then try to parallelize it.
\end{itemize}
\end{frame}

\begin{frame}
\frametitle{Demonstration}
\begin{itemize}
\item Run the sequential hello-uni program.
\item Run the parallel hello-omp program.
\end{itemize}
\end{frame}

\begin{frame}
\frametitle{Implementation}
\begin{itemize}
\item Implement a program that prints 10 lines of ``hello'' with
  OpenMP directive.
\end{itemize}
\end{frame}

\begin{frame}
\frametitle{Discussion}
\begin{itemize}
\item Can you tell how many cores are there in your computer by running
  a OpenMP program?
\end{itemize}
\end{frame}

\subsection{\tt omp\_get\_num\_procs}

\begin{frame}
\frametitle{\tt omp\_get\_num\_procs}
\begin{itemize}
\item Sometimes it is very informative to know the number of cores
  (processing units) in the system.
\item We can call {\tt omp\_get\_num\_procs} to know this information.
\item Note that this is a function, not a compiler directive,
  therefore you need to include {\tt <omp.h>} header file so that the
  compiler can check the prototype.
\end{itemize}
\end{frame}

\begin{frame}
\frametitle{Number of Processors} 
\prototype{omp_get_num_procs}{Get the number of processors}
\end{frame}

\begin{frame}
\frametitle{\tt omp\_get\_num\_threads}
\begin{itemize}
\item It is very informative to know the number of threads in the
  system.
\item We can call {\tt omp\_get\_num\_threads} to know this
  information.
\item Note that this is a function, not a compiler directive.
\end{itemize}
\end{frame}

\begin{frame}
\frametitle{Number of Threads}
\prototype{omp_get_num_threads}{Get the number of threads}
\begin{itemize}
\item Add this function to the hello world to know the number of
  threads running in the system.
\item Compare the number with the number of processors.
\end{itemize}
\end{frame}


\begin{frame}
\frametitle{Get the Number}
\begin{itemize}
\item Now we would like to get the number of threads and cores of our
  system.
\item We simply add the function calls to {\tt hello.c}.
\item Note that we need to include {\tt <omp.h>} header so that the
  compiler can check the function prototype.
\end{itemize}
\end{frame}

% hello-get-num

\begin{frame}
\frametitle{Hello World} \programlistingoption{hello-get-num}{Hello,
  world!}{basicstyle={\tt
    \footnotesize},emph={omp_get_num_threads,omp_get_num_procs,omp}}
\end{frame}

\begin{frame}
\frametitle{Demonstration}
\begin{itemize}
\item Run the parallel hello-get-num-omp program.
\end{itemize}
\end{frame}

\begin{frame}
\frametitle{Discussion}
\begin{itemize}
\item How many threads are reported?
\item How many cores are reported?
\end{itemize}
\end{frame}

\begin{frame}
\frametitle{Problem}
\begin{itemize}
\item Now we would like to get the correct number of threads.
\item The problem with the previous program is that the effect of the
  directive applies only to the next statement, i.e., the printf for
  ``hello, world''.
\item As a result when we call {\tt omp\_get\_num\_threads}, there is
  only one thread left.
\item To solve this problem we use a {\em compound} statement to
  enclose {\em everything} that we would like to run in parallel.
\end{itemize}
\end{frame}

% hello-get-num-all

\begin{frame}
\frametitle{Hello World}
\programlistingoption{hello-get-num-all}{Hello,
  world!}{basicstyle={\tt
    \scriptsize},emph={omp_get_num_threads,omp_get_num_procs}}
\end{frame}

\begin{frame}
\frametitle{Compound Statement}
\begin{itemize}
\item The compound statement has three {\tt printf}.
\item Each of the threads created by the {\tt parallel} directive will
  run all these three statements.
\end{itemize}
\end{frame}

\begin{frame}
\frametitle{Demonstration}
\begin{itemize}
\item Run the parallel hello-get-num-omp program.
\end{itemize}
\end{frame}

\begin{frame}
\frametitle{Discussion}
\begin{itemize}
\item How many threads are reported?
\item How many cores are reported?
\end{itemize}
\end{frame}

\subsection{\tt omp\_set\_num\_thread}

\begin{frame}
\frametitle{In Control}
\begin{itemize}
\item Sometimes it is quite useful to set the number of threads,
  especially after we know the number of cores.
\item The default number of threads is the number of cores, which is
  not necessarily suitable.
\item We can call {\tt omp\_set\_thread\_num} to set the number of
  threads we want to create.
\end{itemize}
\end{frame}


\begin{frame}
\frametitle{Number of Threads}
\prototype{omp_set_num_threads}{Set the number of threads}
\begin{itemize}
\item Set the number of threads.
\end{itemize}
\end{frame}

% hello-get-num-all-set

\begin{frame}
\frametitle{Hello World}
\programlistingoption{hello-get-num-all-set}{Hello,
  world!}{basicstyle={\tt
    \scriptsize},emph={omp_set_num_threads}}
\end{frame}

\begin{frame}
\frametitle{Set the Number}
\begin{itemize}
\item The number of threads is given as the second argument from the
  command line.
\item Then the number of threads is used to call the {\tt
  omp\_set\_num\_threads} function.
\end{itemize}
\end{frame}

\begin{frame}
\frametitle{Demonstration}
\begin{itemize}
\item Run the parallel hello-get-num-set-omp program.
\end{itemize}
\end{frame}

\begin{frame}
\frametitle{Discussion}
\begin{itemize}
\item Does these output follow any time constrain order?
\end{itemize}
\end{frame}

\subsection{\tt omp\_get\_thread\_num}

\begin{frame}
\frametitle{Problem}
\begin{itemize}
\item We cannot identify the threads because their outputs are all the
  same.
\item It will be very useful that each thread can know its identity so
  that they can work accordingly.
\item A thread can call {\tt omp\_get\_thread\_num} to know its index.
\end{itemize}
\end{frame}

\begin{frame}
\frametitle{Thread Number}
\prototype{omp_get_thread_num}{Get thread number}
\end{frame}

% hello-get-num-all-set-show-id

\begin{frame}
\frametitle{Hello World}
\programlistingoption{hello-get-num-all-set-show-id}{Hello,
  world!}{basicstyle={\tt
    \scriptsize},emph={omp_get_thread_num}}
\end{frame}

\begin{frame}
\frametitle{Get the Index}
\begin{itemize}
\item We print the index of a thread by calling {\tt
  omp\_get\_thread\_num}
\item After knowing the index a thread can do things accordingly.
\item The thread number starts with 0.
\end{itemize}
\end{frame}

\begin{frame}
\frametitle{Demonstration}
\begin{itemize}
\item Run the parallel hello-get-num-set-show-id-omp program.
\end{itemize}
\end{frame}

\begin{frame}
\frametitle{Discussion}
\begin{itemize}
\item Does these output follow any time constrain order?
\end{itemize}
\end{frame}

\section{Loop Partition}

\subsection{\tt parallel for}

\begin{frame}
\frametitle{Loop}
\begin{itemize}
\item Now we are ready to partition a loop.
\item Previously we used {\tt parallel} directive to instruct all
  threads to do the {\em same} thing.
\item Now er use {\tt parallel for} directive to ask all threads to
  {\em share} the workload of a for loop.
\end{itemize}
\end{frame}

\begin{frame}
\frametitle{A Parallel For Loop}
\pragma{parallel_for}{Parallel for pragma}
\end{frame}

\begin{frame}
\frametitle{A Parallel For Loop}
\frametitle{Semantics}
\begin{itemize}
\item {\tt parallel for} means that the loop will be distributed among
  all threads for execution.
\item The OpenMP library takes care of all the threading details so
  you do not have to.
\item There are restrictions on the condition of the loop.  
\begin{itemize}
\item The loop iteration can be determined at the compile time.
\end{itemize}
\end{itemize}
\end{frame}

\begin{frame}
\frametitle{Requirement}
\begin{itemize}
\item The initialization must be {\tt var = exp1}.
\item The condition must be {\tt var cond exp2}.
\item The increment must be:
  \begin{itemize}
  \item {\tt var++}, {\tt ++var}.
  \item {\tt var--}, {\tt --var}.
  \item {\tt var = var + exp3}, {\tt var = var - exp3}.
  \item {\tt var += exp3}, {\tt var -= exp3}.
  \end{itemize}
\end{itemize}
\end{frame}

\begin{frame}
\frametitle{Steps}
\begin{itemize}
\item Get the number of threads from the second command line argument.
\item Get the number of for loop iterations from the third command
  line argument.
\item Add the {\tt parallel for} pragma to the loop.
\item That is it!
\end{itemize}
\end{frame}

% for 

\begin{frame}
\frametitle{Loop} 
\programlistingoption{for}{A for loop}{basicstyle={\tt
    \scriptsize},emph={parallel,for}}
\end{frame}

\begin{frame}
\frametitle{Synchronization}
\begin{itemize}
\item Note that all threads, except the main thread, will only be
  active during the loop.  
\item All threads partitioning the loop will synchronize with each
  other, i.e., do a barrier synchronization, before leaving the loop.
\item This {\em join} is automatic and we do not need to program
  anything.
\end{itemize}
\end{frame}


\begin{frame}
\frametitle{Demonstration}
\begin{itemize}
\item Run the parallel for-omp program.
\end{itemize}
\end{frame}

\begin{frame}
\frametitle{Discussion}
\begin{itemize}
\item Observe the output of for-omp and describe the rule about which
  thread executes which loop iteration.
\end{itemize}
\end{frame}

\begin{frame}
\frametitle{$n$ Queen with OpenMP}
\begin{itemize}
\item Now we want to parallelize the $n$-queen program previously
  written for Pthread.
\item We only need to parallel the loop in {\tt main}, in which each
  iteration calls the {\tt queen} we wrote previously.
\item We can write the sequential version first, then parallelize it
  with OpenMP directives.
\end{itemize}
\end{frame}

% queen

\begin{frame}
\frametitle{\tt Declaration} \programlistingoption{queen}{Declaration}{linerange={begin-ok},emph={n}}
\end{frame}

\begin{frame}
\frametitle{\tt ok} \programlistingoption{queen}{\tt
  main}{linerange={ok-queen},emph={position},basicstyle={\tt \small}}
\end{frame}

\begin{frame}
\frametitle{\tt queen} \programlistingoption{queen}{\tt
  main}{linerange={queen-main},emph={position,next}}
\end{frame}

\begin{frame}
\frametitle{\tt main} \programlistingoption{queen}{\tt
  main}{linerange={main-end},emph={parallel,for,position},basicstyle={\tt \footnotesize}}
\end{frame}

\begin{frame}
\frametitle{\tt main}
\begin{itemize}
\item We only need to parallel the loop in {\tt main} by adding
 {\tt parallel for} directive.
\item Each thread will call {\tt queen} with {\tt position[0]} set to
  different {\tt i}, as we observed in the previous loop partition
  example.
\end{itemize}
\end{frame}

\begin{frame}
\frametitle{Demonstration}
\begin{itemize}
\item Run and time the sequential queen-uni program.
\item Run and time the parallel queen-omp program.
\end{itemize}
\end{frame}

\begin{frame}
\frametitle{Discussion}
\begin{itemize}
\item Does the sequential program produce correct answer?
\item Does the parallel program produce correct answer?
\end{itemize}
\end{frame}

% queen-private

\subsection{\tt private}

\begin{frame}
\frametitle{$n$ Queen with OpenMP}
\begin{itemize}
\item The problem with the previous program is that all threads share
  the {\tt position} array.
\item We now want to ask OpenMP to create a {\em private} variable of
  {\tt position} in each thread, so they can produce the correct
  answer.
\item We simply add a {\tt private} clause to do this.  
\item Note that the private variable has nothing to do with the global
  variable -- they are completely independent.
\end{itemize}
\end{frame}

\begin{frame}
\frametitle{Private Clause}
\prototype{private}{Private clause}
\begin{itemize}
\item A private clause is an optional component of a pragma.
\item We can add this clause to a {\tt parallel for} pragma.
\item Note that the initial value of the private variable position
 has nothing to do with the global variable position.
\end{itemize}
\end{frame}

% queen-private 

\begin{frame}
\frametitle{\tt main} \programlistingoption{queen-private}{\tt
  main}{linerange={main-end},emph={private,position},basicstyle={\footnotesize \tt}}
\end{frame}

\begin{frame}
\frametitle{Demonstration}
\begin{itemize}
\item Run and time the sequential queen-private-uni program.
\item Run and time the parallel queen-private-omp program.
\item Calculate the speedup.
\end{itemize}
\end{frame}

\begin{frame}
\frametitle{Implementation}
\begin{itemize}
\item Implement a program that solves the $n$ queen problem with
  OpenMP directive.
\end{itemize}
\end{frame}

\begin{frame}
\frametitle{Discussion}
\begin{itemize}
\item Does the sequential program produce correct answer?
\item Does the parallel program produce correct answer?
\end{itemize}
\end{frame}

% for-private

\begin{frame}
\frametitle{Private Variable}
\begin{itemize}
\item We use the following example to emphasize that a private
  variable has nothing to do with the global variable.
\item One can think of the private variable as a new variable declared
  with the thread with the same name.
\end{itemize}
\end{frame}

\begin{frame}
\frametitle{\tt parallel for} \programlistingoption{for-private}{\tt
  main}{linerange={main-end},basicstyle={\tt \scriptsize},emph={private,v,101}}
\end{frame}

\begin{frame}
\frametitle{Demonstration}
\begin{itemize}
\item Run for-private-omp and observe the output.
\end{itemize}
\end{frame}

\begin{frame}
\frametitle{Independent variables}
\begin{itemize}
\item We can clearly see that the values of the private {\tt v} have
  nothing to do with the value of the global {\tt v}.
\item The value of the global {\tt v} is not affected by any
  operations within the loop.
\end{itemize}
\end{frame}

\begin{frame}
\frametitle{Private Variables}
\centerline{\pgfimage[width=0.9\textwidth]{private.pdf}}
\end{frame}

\begin{frame}
\frametitle{Discussion}
\begin{itemize}
\item Check queen-private.c and make sure that you understand that
  even though the global position and private position are
  independent, the code still works.
\item What will happen if we declare {\tt position} {\em within} the
  loop?
\end{itemize}
\end{frame}



\begin{frame}
\frametitle{Sum}
\begin{itemize}
\item We now want to compute the total number of solutions.
\item We use a variable {\tt num} to store the number of solutions
  from one iteration, and a variable {\tt numSolution} to store the
  total number of solutions.
\item The parallel for directive applies only to the for loop, so the
  printf after the loop will only be run by the main thread.
\end{itemize}
\end{frame}

% queen-private-sum

\begin{frame}
\frametitle{\tt main} \programlistingoption{queen-private-sum}{\tt
  main}{linerange={loop-end},emph={num,numSolution},basicstyle={\scriptsize \tt}}
\end{frame}

\begin{frame}
\frametitle{Demonstration}
\begin{itemize}
\item Run the sequential queen-private-sum-uni program.
\item Run the parallel queen-private-sum-omp program.
\end{itemize}
\end{frame}


\begin{frame}
\frametitle{Discussion}
\begin{itemize}
\item Does the sequential program produce correct answer?
\item Does the parallel program produce correct answer?
\end{itemize}
\end{frame}


\begin{frame}
\frametitle{Problem}
\begin{itemize}
\item The problem with the previous program is that we forgot to
  declare {\tt num} as private, so all threads use the same copy.
\item We simply add {\tt num} to the private clause and fix the
  problem.
\item Note that the {\tt numSolution} needs to remain global since it
  is the total number.
\end{itemize}
\end{frame}

% queen-private-sum-num

\begin{frame}
\frametitle{\tt main} \programlistingoption{queen-private-sum-num}{\tt
  main}{linerange={loop-end},emph={private,num},basicstyle={\footnotesize \tt}}
\end{frame}

\begin{frame}
\frametitle{Demonstration}
\begin{itemize}
\item Run and time the sequential queen-private-sum-uni program.
\item Run and time the parallel queen-private-sum-omp program.
\item Calculate the speedup.
\end{itemize}
\end{frame}

\begin{frame}
\frametitle{Discussion}
\begin{itemize}
\item Does the sequential program produce correct answer?
\item Does the parallel program produce correct answer?
\item Does the parallel program {\em always} produce correct answer?
\end{itemize}
\end{frame}



\begin{frame}
\frametitle{Problem}
\begin{itemize}
\item To illustrate that there is also a race condition in the
  previous program we ``amplify'' it by adding to a global variable
  {\tt numSolution}.
\item Every time {\tt queen} finds a solution, it adds 1 to {\tt
  numSolution}.
\item Finally the main program prints {\tt numSolution}
\end{itemize}
\end{frame}

% queen-private-sum-race

\begin{frame}
\frametitle{\tt Declaration} \programlistingoption{queen-private-sum-race}{\tt
  main}{linerange={begin-ok},emph={numSolution}}
\end{frame}

\begin{frame}
\frametitle{\tt queen} \programlistingoption{queen-private-sum-race}{\tt
  main}{linerange={queen-main},emph={numSolution}}
\end{frame}

\begin{frame}
\frametitle{\tt main} \programlistingoption{queen-private-sum-race}{\tt
  main}{linerange={loop-end},emph={numSolution},basicstyle={\tt \small}}
\end{frame}

\begin{frame}
\frametitle{Demonstration}
\begin{itemize}
\item Run the sequential queen-private-sum-uni program.
\item Run the parallel queen-private-sum-omp program.
\end{itemize}
\end{frame}

\begin{frame}
\frametitle{Discussion}
\begin{itemize}
\item Does the sequential program produce correct answer?
\item Does the parallel program produce correct answer?
\end{itemize}
\end{frame}



\begin{frame}
\frametitle{Problem}
\begin{itemize}
\item To avoid the extremely rare race condition in {\tt
  queen-private-sum} we use an array {\tt num} to store the number of
  solutions from each subtree.
\item Later we sequentially sum the numbers in {\tt num} to get the
  correct total number of solutions.
\end{itemize}
\end{frame}

% queen-private-sum-array

\begin{frame}
\frametitle{\tt main} \programlistingoption{queen-private-sum-array}{\tt
  main}{linerange={loop-end},emph={num,numSolution},basicstyle={\tt \footnotesize}}
\end{frame}

\begin{frame}
\frametitle{Demonstration}
\begin{itemize}
\item Run and time the sequential queen-private-sum-array-uni program.
\item Run and time the parallel queen-private-sum-array-omp program.
\end{itemize}
\end{frame}

\begin{frame}
\frametitle{Discussion}
\begin{itemize}
\item Does the sequential program produce correct answer?
\item Does the parallel program produce correct answer?
\end{itemize}
\end{frame}

\subsection{\tt reduction}

\begin{frame}
\frametitle{Reduction}
\begin{itemize}
\item It is very common that we need to collect answers from all
  threads and combine them.
\item OpenMP provides a {\em simple} mechanism called reduction to
  derive the final answer by combining many values into one.
\end{itemize}
\end{frame}


\begin{frame}
  \frametitle{Reduction Clause} \prototype{reduction}{Reduction
    clause of a {\tt parallel for} pragma}
  \begin{description}
  \item[operation] Operation to be performed on the partial answers.
  \item[variable] The partial answer the operation to be perform
    on.
  \end{description}
\end{frame}

\begin{frame}
  \frametitle{Reduction Operation} 
  The reduction supports the following operations.
  \begin{description}
  \item [+] Sum
  \item [*] Product
  \item [\&] Bitwise and
  \item [$|$] Bitwise or
   % \item \verb+^+ Bitwise exclusive or
  \item [\&\&] Logical and
  \item [$||$] Logical or
  \end{description}
\end{frame}

% queen-private-sum-reduction

\begin{frame}
\frametitle{\tt main}
\programlistingoption{queen-private-sum-reduction}{\tt
  main}{linerange={loop-end},emph={num,reduction,numSolution},basicstyle={\tt
    \footnotesize}}
\end{frame}

\begin{frame}
\frametitle{Notes}
\begin{itemize}
\item The variable for reduction {\tt numSolution} is {\em private}.
\item We want to compute the total number of solutions so we use {\tt
  +} operator.
\item The variable {\tt num} is declared within the loop so it is
  private to its thread.
\end{itemize}
\end{frame}

\begin{frame}
\frametitle{Demonstration}
\begin{itemize}
\item Run the sequential queen-private-sum-reduction-uni program.
\item Run the parallel queen-private-sum-reduction-omp program.
\end{itemize}
\end{frame}

\begin{frame}
\frametitle{Discussion}
\begin{itemize}
\item Does the sequential program produce correct answer?
\item Does the parallel program produce correct answer?
\item Can you derive the characteristic of the operations supported by
  reduction?  What do they have in common?
\end{itemize}
\end{frame}

% firstprivate

\subsection{\tt firstprivate}

\begin{frame}
\frametitle{Communication}
\begin{itemize}
\item Remember that the global variable and the private variables are
  different things.
\item In the previous program we initialize the global variable {\tt
  numSolution} to 0, but how do all private variables {\tt
  numSolution} knows this initial value???
\item We need to concept of {\tt firstprivate}.
\end{itemize}
\end{frame}

\begin{frame}
\frametitle{\tt firstprivate}
\begin{itemize}
\item Usually the private and global variables are different things.
\item Sometimes we want to ``pass'' the value of the global variable
  to threads as the {\em initial} values of their private variables.
\item We can use a {\tt firstprivate} clause to do this.
\end{itemize}
\end{frame}

\begin{frame}
\frametitle{\tt firstprivate} \programlistingoption{for-private-first}{\tt
  main}{linerange={main-end},basicstyle={\tt \scriptsize},emph={firstprivate}}
\end{frame}

\begin{frame}
\frametitle{\tt firstprivate}
\begin{itemize}
\item The first command line argument specifies the number of iterations.
\item The second command line argument specifies the number of threads.
\item The value of global {\tt v} is used to initialize the private
  {\tt v} when the threads starts.
\item Note that the initialization only happens when a thread starts,
  not every loop iteration.
\end{itemize}
\end{frame}

\begin{frame}
\frametitle{Demonstration}
\begin{itemize}
\item Run the loop-private-first-omp program with four threads and
  eight iterations, and observe the output.
\end{itemize}
\end{frame}

\begin{frame}
\frametitle{First Private Variables}
\centerline{\pgfimage[width=0.9\textwidth]{first-private.pdf}}
\end{frame}

\begin{frame}
\frametitle{Communication}
\begin{itemize}
\item {\tt firstprivate} provides a mechanism to pass information
  into threads.
\item We can also do this by placing the information into a global
  variable.  
\item However, if the information will be manipulated and modified by
  each thread individually, then we can use {\tt firstprivate} to make
  a private copy within each thread.  This is very convenient.
\end{itemize}
\end{frame}

\begin{frame}
\frametitle{Discussion}
\begin{itemize}
\item Describe when and the number of times the private variables are
  initialized.
\item Describe how to implement the part of a reduction that we need
  to pass the initial values of {\tt numSolution} in {\tt queen-private-sum-redction.c}.
\end{itemize}
\end{frame}

% sudoku.c

\begin{frame}
\frametitle{Sudoku}
\centerline{\pgfimage[height=0.8\textheight]{sudoku}}
\end{frame}

\begin{frame}
\frametitle{Sudoku}
\begin{itemize}
\item There is a 9 by 9 board with some numbers from 1 to 9 already in
  it (in black).
\item Place numbers from 1 to 9 (in red) into empty squares, so that
  the every row, every column, and all nine 3 by 3 squares will have
  numbers 1 to 9.
\item We would like to compute the total number of solutions.
\end{itemize}

\end{frame}


\begin{frame}
\frametitle{main} \programlistingoption{sudoku}{\tt
  main}{linerange={main-omp},basicstyle={\tt \small},emph={firstprivate,sudoku,firstZero}}
\end{frame}

\begin{frame}
\frametitle{Solution}
\begin{itemize}
\item We represent the board by a 9 by 9 array of integers {\tt
  sudoku}.
\item The input has all the initial numbers in the board, where an 0
  is an empty cell.
\item We need to remember the index of the first zero (\tt firstZero)
  so that we can start the recursion from there.
\end{itemize}
\end{frame}


\begin{frame}
\frametitle{main} \programlistingoption{sudoku}{\tt
  main}{linerange={omp-end},basicstyle={\tt \footnotesize},emph={firstprivate,sudoku,firstZero,_OPENMP,omp_set_num_threads}}
\end{frame}

\begin{frame}
\frametitle{Compile}
\begin{itemize}
\item We explicitly set the number of threads to 9 for the OpenMP
  version.  
\item However, we also want to use to same code to compiler a
  sequential version. 
\item The solution is test if {\tt \_OPENMP} is defined, which means
  the {\tt -fopenmp} flag is used, then we will compiler the {\tt
    omp\_set\_num\_threads} call.
\item You may use {\tt -E} to verify that the correct code is
  compiled.
\end{itemize}
\end{frame}

\begin{frame}
\frametitle{Solution}
\begin{itemize}
\item After we read the inputs into {\tt sudoku}, we then use {\tt
  firstprivate} to send the input to all threads.
\item We use nine iterations to compute the total number of solutions,
  where the $i$-th iteration computes the number of solutions if we
  place he first empty cell with $i$.
\item An iteration first tries to place a number into the first empty
  cell if there is no conflict, then call {\tt placeNumber} to compute
  the number of solutions.
\item We use a reduction on the variable {\tt numSolution}, which is
  the number of solutions found by a recursive function {\tt
    placeNumber}.
\end{itemize}
\end{frame}

\begin{frame}
\frametitle{\tt conflict} \programlistingoption{sudoku}{\tt
  conflict}{linerange={conflict-placeNumber},basicstyle={\tt
    \scriptsize},emph={rowColConflict,blockConflict}}
\end{frame}

\begin{frame}
\frametitle{\tt rowColConflict} \programlistingoption{sudoku}{\tt
  rowColConflict}{linerange={rowColConflict-blockConflict},basicstyle={\tt
    \scriptsize},emph={rowColConflict,blockConflict}}
\end{frame}

\begin{frame}
\frametitle{\tt blockConflict} \programlistingoption{sudoku}{\tt
  blockConflict}{linerange={blockConflict-conflict},basicstyle={\tt
    \scriptsize},emph={rowColConflict,blockConflict}}
\end{frame}

\begin{frame}
\frametitle{\tt placeNumber}
\begin{itemize}
\item The function {\tt placeNumber} tries placing a number at the
  $n$-th position on the board {\tt sudoku}, where {\tt n} goes from 0
  to 80.
\item If {\tt n} is 81 then we have placed all numbers and a solution
  is found.
\item Otherwise we determine the row and column of this cell.  If this
  cell has a number, then we skip it since it is a part of the input.
\end{itemize}
\end{frame}


\begin{frame}
\frametitle{\tt placeNumber} \programlistingoption{sudoku}{\tt
  placeNumber}{linerange={placeNumber-numSolution},basicstyle={\tt \small},emph={sudoku,placeNumber}}
\end{frame}

\begin{frame}
\frametitle{\tt placeNumber}
\begin{itemize}
\item If there is no number in the cell, then we check if we can place
  a number {\tt try} into this cell by calling {\tt conflict}.  If we
  can then we go to the next level of recursion.
\item Since no conflict was found we place the number into the cell
  and call {\tt placeNumber} to go for the next cell.
\item After we tried all numbers from 1 to 9, we report the number of
  solutions we found.
\item Note that we need to reset the cell back to 0 so later recursion
  will not mistake it for being an input.
\end{itemize}
\end{frame}


\begin{frame}
\frametitle{\tt placeNumber} \programlistingoption{sudoku}{\tt
  placeNumber}{linerange={numSolution-main},basicstyle={\tt
    \footnotesize},emph={numSolution,placeNumber}}
\end{frame}

\begin{frame}
\frametitle{Demonstration}
\begin{itemize}
\item Run and time the sudoku-uni program.
\item Run and time the sudoku-omp program.
\item Compute the speedup.
\end{itemize}
\end{frame}

\begin{frame}
\frametitle{Discussion}
\begin{itemize}
\item Do you think the sudoku program balance the workload evenly?
  Explain your answer.
\item Is there any way to balance the workload better?
\end{itemize}
\end{frame}

\subsection{\tt lastprivate}

% loop-private-first-last

\begin{frame}
\frametitle{Communication}
\begin{itemize}
\item Sometimes we do need to pass the value of one private variable
  back to its global counterpart.
\item The {\tt lastprivate} clause will pass the value of private
  variable of the thread which runs the {\em last} iteration.
\item Note that this is not the thread that runs last -- it is the
  thread that runs the {\em last} iteration.
\end{itemize}
\end{frame}

\begin{frame}
\frametitle{Both {\tt firstprivate} and {\tt lastprivate}}
\centerline{\pgfimage[width=0.9\textwidth]{first-private-last.pdf}}
\end{frame}

\begin{frame}
\frametitle{\tt lastprivate} \programlistingoption{for-private-first-last}{\tt
  main}{linerange={v-end},basicstyle={\tt \footnotesize},emph={firstprivate,lastprivate}}
\end{frame}


\begin{frame}
\frametitle{Demonstration}
\begin{itemize}
\item Run the loop-private-first-last-omp program with four threads
  and eight iterations, and observe the output.
\end{itemize}
\end{frame}

\begin{frame}
\frametitle{Discussion}
\begin{itemize}
\item Describe a scenario that {\tt lastprivate} will be useful.
\end{itemize}
\end{frame}

\section{Critical Section}

\begin{frame}
  \frametitle{Critical Section}
  \begin{itemize}
  \item It is very common that we need to implement a critical section
    that only one thread can access at a time.
  \item OpenMP provides a {\em critical} pragma to implement a critical
    section.  
  \item A {\tt critical} directive will make the following statement a
    critical section.
  \end{itemize}
\end{frame}

\subsection{\tt critical}

\begin{frame}
  \frametitle{Critical Pragma} 
  \prototype{critical}{Critical pragma}
\end{frame}

% queen-private-sum-race-critical

\begin{frame}
\frametitle{\tt queen}
\programlistingoption{queen-private-sum-race-critical}{\tt
  main}{linerange={queen-main},emph={numSolution,omp,critical},basicstyle={\tt
    \footnotesize}}
\end{frame}

\begin{frame}
\frametitle{\tt main}
\programlistingoption{queen-private-sum-race-critical}{\tt
  main}{linerange={loop-end},emph={numSolution},basicstyle={\tt
    \footnotesize}}
\end{frame}

\begin{frame}
\frametitle{Notes}
\begin{itemize}
\item Whenever {\tt queen} finds a solution, it adds 1 to {\tt
  numSolution}.
\item This addition is a critical section because we put a {\tt
  \#pragma omp critical} in front of it.
\item If you want to place multiple statements into a critical
  section, then you need to put them into a compound statement.
\end{itemize}
\end{frame}

\begin{frame}
\frametitle{Demonstration}
\begin{itemize}
\item Run the sequential queen-private-sum-critical-uni program.
\item Run the parallel queen-private-sum-critical-omp program.
\end{itemize}
\end{frame}

\begin{frame}
\frametitle{Discussion}
\begin{itemize}
\item Does the sequential program produce correct answer?
\item Does the parallel program produce correct answer?
\item Time and compare the correct solutions you know so far.
\end{itemize}
\end{frame}



\begin{frame}
\frametitle{The Big Question} {\huge Do you think it is easier to
  write OpenMP program than Pthread program?}
\end{frame}


\end{document}

