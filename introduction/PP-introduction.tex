\documentclass{beamer}
\usetheme{Warsaw}

\usepackage{color}
\usepackage{CJK}
\usepackage{listings}
\usepackage{url}
\usepackage{booktabs}
\usepackage{pgf}
\usepackage{multicol}


% global setting

\newcommand{\comment}[1]{}

% list environment gloabal setting

% to specify the size of fonts in lst environment

\lstset{showstringspaces=false}
\lstset{language=C}
\lstset{basicstyle=\tt}
\lstset{keywordstyle=\color{blue}}
\lstset{emphstyle=\color{red}}
\lstset{numbers=left}
\lstset{frame=single}
\lstset{rangeprefix=\/*\ }
\lstset{rangesuffix=\ *\/}
\lstset{includerangemarker=false}

%% slides related macros

\newenvironment{mytitle}[2]
{\begin{frame}{#1} \vspace{0.2\textheight}{\huge #2} \end{frame}}

%% Macro for non-numbered frames listing 
%% \nonumberlisting{title}{file}{options}

\newcommand{\nonumberlisting}[3]
{\noindent{#1}
{\lstinputlisting[#3]{#2}}}

%% Program listing macros

%% \programlisting{file}{desp}{font}
%% \programlistingfirst{file}{desp}{font}{start}{end}
%% \programlistingrest{file}{descp}{font}{start}{end}


%% List a program in one page with a font size.

\newcommand{\programlisting}[3]
{\renewcommand{\lstlistingname}{#3 \bf Example}
\lstinputlisting[basicstyle={#3 \tt},caption={#3 \bf (#1) #2}]{#1}}

%% List a program for the first page within a given range.

\newcommand{\programlistingfirst}[5]
{\renewcommand{\lstlistingname}{#3 \bf Example}
\lstinputlisting[basicstyle={#3 \tt},linerange={#4-#5},caption={#3 \bf (#1) #2}]{#1}}

%% List a program for the second page (or later).

\newcommand{\programlistingrest}[5]
{\lstinputlisting[basicstyle={#3 \tt},linerange={#4-#5}]{#1}}

%% List a program in two pages.

\newcommand{\programlistingtwoslides}[6]
{\begin{frame}                  % do not add [fragile]
\programlistingfirst{#1}{#2}{#3}{#4}{#5}
\end{frame}
\begin{frame}
\programlistingrest{#1}{#2}{#3}{#5}{#6}
\end{frame}}

%% List a program in three pages.

\newcommand{\programlistingthreeslides}[7]
{\begin{frame}                  % do not add [fragile]
\programlistingfirst{#1}{#2}{#3}{#4}{#5}
\end{frame}
\begin{frame}
\programlistingrest{#1}{#2}{#3}{#5}{#6}
\end{frame}
\begin{frame}
\programlistingrest{#1}{#2}{#3}{#6}{#7}
\end{frame}
}


%% Program input/output macros

%% \programoutput{file}
%% \programoutputfirst{file}{font}{options}
%% \programoutputrest{file}{font}{startline}{endline}
%% \programoutputtwoslides{file}{font}{start}{endl}{start2}{end2}

%% Program output macor with default setting

\newcommand{\programoutput}[1]
{\programoutputfirst{#1}{}{}}

%% Program output macor for the first page

\newcommand{\programoutputfirst}[3]
{\lstset{basicstyle={#2 \tt}}
\nonumberlisting{#2 \bf 輸出}{#1.out}{#3}}

%% Program output macor for later page

\newcommand{\programoutputrest}[4]
{\lstinputlisting[firstnumber={#3},basicstyle={\tt #2},linerange={#3-#4}]
{#1.out}}

%% Program output macor for two slides

\newcommand{\programoutputtwoslides}[6]
{\begin{frame}                  % do not add [fragile]
\programoutputfirst{#1}{#2}{linerange={#3-#4}}
\end{frame}
\begin{frame}
\programoutputrest{#1}{#2}{#5}{#6}
\end{frame}}


%% \programinput{file}
%% \programinputfirst{file}{font}{option}

%% Program input macor with default setting

\newcommand{\programinput}[1]
{\programinputfirst{#1}{}{}}

%% Program input macor for the first page

\newcommand{\programinputfirst}[3]
{\lstset{basicstyle={#2 \tt}}
\nonumberlisting{#2 \bf 輸入}{#1.in}{#3}}


%% \piotwocol{file}

\newcommand{\piotwocol}[1]{\piotwocoldetail{#1}{}{}}

\newcommand{\piotwocoldetail}[2]{
\begin{columns}[t,onlytextwidth]
\begin{column}{0.45\textwidth}
\programinputfirst{#1}{#2}{}
\end{column}
\begin{column}{0.45\textwidth}
\programoutputfirst{#1}{#2}{}
\end{column}
\end{columns}}

%% \piotworow{file}

\newcommand{\piotworow}[1]
{\programinput{#1}
\programoutput{#1}}

\newcommand{\piotworowdetail}[2]
{\programinputfirst{#1}{#2}{}
\programoutputfirst{#1}{#2}{}}


%% \newcommand{\headerfile}[1]
%% {\nonumberlisting {{\bf 標頭檔} {\tt #1}}{#1}{}}

\newcommand{\header}[2]{\headerdetail{#1}{#2}{}}

\newcommand{\headerdetail}[3]
{{\renewcommand{\lstlistingname}{#3 \bf Header}
\lstset{basicstyle={#3 \tt}}
\lstset{label=header:#1,caption={(#1) #2}}
\lstinputlisting{#1}}}

%% Macro for prototype
%% \prototype{file}

\newcommand{\prototype}[1]{\prototypedetail{#1}{}}

%% Macro for prototype with detialed setting
%% \prototypedetail{file}{font}

\newcommand{\prototypedetail}[2]
{{\renewcommand{\lstlistingname}{#2 \bf Prototype}
\lstset{basicstyle={#2 \tt}}
\lstset{frame=single,label=prototype:#1,caption={#2}}
\lstinputlisting{#1}}}

%% Macro for phrase
%% \nphrase{file}{desc}

%% \newcommand{\nphrase}[2]
%% {{\renewcommand{\lstlistingname}{\lstfontsize \bf 片語}
%% \lstset{label=phr:#1,frame=single,caption={\lstfontsize #2}}
%% \lstset{basicstyle={\lstfontsize \tt}}
%% \lstinputlisting{#1.p}}
%% }

\newcommand{\nphrase}[2]
{\nphrasedetail{#1}{#2}{}}

%% Macro for detail phrase
%% \nphrasedetail{file}{desc}{font}

\newcommand{\nphrasedetail}[3]
{{\renewcommand{\lstlistingname}{#3 \bf 片語}
\lstset{label=phr:#1,frame=single,caption={#3 #2}}
\lstset{basicstyle={#3 \tt}}
\lstinputlisting{#1.p}}}



\newcommand{\inputfile}[1]
{\nonumberlisting {{\bf 輸入檔} {\tt #1}}{#1}{}}

\newcommand{\outputfile}[1]
{\nonumberlisting {{\bf 輸出檔} {\tt #1}}{#1}{}}

% Macro for listings that are labeled.
% [1] is the filename, [2] is the description

\newcommand{\commandline}[1]
{\lstset{basicstyle={\tt}}
\nonumberlisting{\bf 命令列}{#1.bat}{}}

% {\boxedlisting{\bf 命令列}{#1.bat}}

% Macro for listings that are labeled.
% [1] is the filename, [2] is the description

\newcommand{\out}[2]
{{\renewcommand{\lstlistingname}{\bf 執行結果}
\lstset{label=out:#1,caption=#2}
\lstinputlisting{#1.out}}}






\begin{document}
\begin{CJK}{UTF8}{bsmi}

  \title{Introduction to Parallel Computing}

  \author{Pangfeng Liu \\ National Taiwan University}

  \begin{frame}
    \titlepage
  \end{frame}

  \section{Motivation} 
  \begin{frame}
    \frametitle{Why Parallel Computing?}
    \begin{itemize}
    \item Solve a problem faster.
    \item Solve a problem better.
    \end{itemize}
  \end{frame}

  \begin{frame}
    \frametitle{Olympic Games} The modern Olympic Games are the leading international sporting event featuring summer and winter sports competitions in which thousands of athletes from around the world
    participate in a variety of competitions.\footnote{\url{http://en.wikipedia.org/wiki/Olympic_Games}}

  \end{frame}

  \begin{frame}
    \frametitle{Olympic Motto}
    The Olympic motto, Citius, Altius, Fortius, a Latin expression meaning
    ``Faster, Higher, Stronger'' was proposed by Pierre de Coubertin in
    1894 and has been official since 1924.
  \end{frame}

  \begin{frame}
    \frametitle{Faster}
    \begin{itemize}
    \item The reigning 100 m Olympic champion is often named ``the fastest
      man/woman in the world''.
    \item The world record is 9.58 seconds.
    \end{itemize}
  \end{frame}

  \begin{frame}
    \frametitle{Higher}
    \begin{itemize}
    \item The high jump is a track and field event in which competitors
      must jump over a horizontal bar placed at measured heights without the aid of certain devices.
    \item The world record is 2.45m.
    \end{itemize}
  \end{frame}

  \begin{frame}
    \frametitle{Stronger}
    \begin{itemize}
    \item Olympic-style weightlifting is an athletic discipline in the modern Olympic program in which the athlete attempts a maximum-weight single lift of a barbell loaded with weight plates.
    \item The world record is 305 kg, the sum of snatch and clean \& jerk. 
    \end{itemize}
  \end{frame}

  \begin{frame}
    \frametitle{Discussion}
    \Large 
    \begin{itemize}
    \item Why faster, higher, and stronger?
    \end{itemize}
  \end{frame}

  %% \begin{frame}
  %% \frametitle{Question}
  %% What are the purposes of parallel computing?
  %% \begin{itemize}
  %% \item Solve a problem faster.
  %% \item Solve a problem better.
  %% \item Solve a problem cheaper.
  %% \item Solve a problem greener.
  %% \end{itemize}
  %% \end{frame}

  \subsection{Faster}

  \begin{frame}
    \frametitle{Why Faster?}
    \begin{itemize}
    \item Many computations are {\em slow}.
    \item Many computations are {\em time critical}.
    \end{itemize}
  \end{frame}

  \begin{frame}
    \frametitle{Slow Computation}
    \begin{itemize}
    \item Brute force search
    \item Computer simulation
    \item Exponential time complexity
    \item Grand Challenge problems
    \end{itemize}
  \end{frame}

  \begin{frame}
    \frametitle{Traveling Salesman} Given a set of cities and the distance
    between two cities, find the shortest route that goes through all
    cities without visiting any city twice.
    \begin{itemize}
    \item A famous NP-complete problem, i.e., a proven hard-as-hell problem in computer science.
    \item Easy, you permute all cities and find the shortest route.
    \item The number of permutations is $(n-1)!$ where $n$ is the
      number of cities.
    \end{itemize}
  \end{frame}

  \begin{frame}
    \frametitle{Factorial}
    \begin{itemize}
    \item Consider a traveling salesman problem with 101 cities.
    \item $100!$ is roughly $9.332621544 \times 10^{157}$.
    \item Assume that the computation of the length of a path can be done
      in 100 cycles.
    \item Assume that a computer runs at 10 GHz.
    \item It takes $9.332621544 \times 10^{146}$ seconds to enumerate all
      paths.
    \end{itemize}
  \end{frame}

  \begin{frame}
    \frametitle{Computation Time}
    \begin{itemize}
    \item A year has 31,556,926 seconds, so the computation takes $2.9574
      \times 10^{130}$ billion years.
    \item The age of earth is 4.54 billion
      years\footnote{\url{http://en.wikipedia.org/wiki/Age_of_the_Earth}}.
    \item The age of the universe is 13.798 billion
      years\footnote{\url{http://en.wikipedia.org/wiki/Age_of_the_universe}}.
    \item Do you honestly think anyone can live to see the results?
    \end{itemize}
  \end{frame}

  \begin{frame}
    \frametitle{Algorithm Optimization} Despite techniques like
    branch-and-bound and $A^*$ search can reduce the number of cases one
    needs to examine, the sheer number of permutations is enormous.
  \end{frame}

  \begin{frame}
    \frametitle{Discussion} \Large 
    \begin{itemize}
    \item Give an example of computation that will take a lot of time.
    \end{itemize}
  \end{frame}

  \begin{frame}
    \frametitle{Time Critical Computation}
    \begin{itemize}
    \item Weather forecast
    \item Stock market
    \item Radar signal processing
    \end{itemize}
  \end{frame}

  \begin{frame}
    \frametitle{Worried? Me?}
    \begin{itemize}
    \item We do not need to do anything because computers become faster
      {\em automatically}!
    \item ``Moore's law'' is the observation that, over the history of computing hardware, the number of transistors in dense integrated circuit doubles approximately every two years\footnote{\url{http://en.wikipedia.org/wiki/Moore\%27s_law}}.
    \item If the speed of a CPU is proportional to the number of transistors, we expect an eight times speed improvement in six years.
    \end{itemize}
  \end{frame}

  \begin{frame}
    \frametitle{Worried? Yes.}
    \begin{itemize}
    \item We assume we have an $O(n^3)$ ``efficient'' algorithm to solve a
      problem, e.g., matrix multiplication.
    \item You can solve problem size twice as large as the original one if
      you wait six years.
    \item By the time you wait for the hardware to catch up, your career
      (e.g., as a Ph.D. student) is over.
    \end{itemize}
  \end{frame}

  \begin{frame}
    \frametitle{How to be Faster?}  \Huge Hardware improvement cannot help
    you, OK!! It would be best if you had better solutions than waiting.
  \end{frame}

  \begin{frame}
    \frametitle{Late Information}
    \begin{itemize}
    \item Late is {\em not} better than nothing.
    \item Justice delayed is justice denied.
    \item A weather forecast for tomorrow takes three days.
    \item A stock recommendation for next week takes three weeks.
    \item The computation of an early warning radar system takes three
      times for an enemy missile to destroy the radar.
    \end{itemize}
  \end{frame}

  \begin{frame}
    \frametitle{How to be Faster?}
    \Huge Having more than one CPU to work on the problem seems to be a
    reasonable choice.
  \end{frame}

  \begin{frame}
    \frametitle{Discussion} \Large 
    \begin{itemize}
    \item Give an example of time-critical computation.
    \item Describe Moore's Law.
    \end{itemize}
  \end{frame}


  %% \begin{frame}
  %% \frametitle{Question}
  %% For a traveling salesman problem of 5 cities, how many paths a brute
  %% force search needs to examine in order to determine the shortest path?
  %% \begin{itemize}
  %% \item 5
  %% \item 20
  %% \item 24
  %% \item 120
  %% \end{itemize}
  %% \end{frame}

  \subsection{Better}

  \begin{frame}
    \frametitle{Why Better?}
    \begin{itemize}
    \item People are {\em greedy}.
    \item Video quality is an increasing function of time.
      \begin{itemize}
      \item VCD (NTSC) $352 \times 240$.
      \item DVD (25 frame rate) $720 \times 576$.
      \item Blueray (HD) $1920 \times 1080$.
      \end{itemize}
    \item The number of pixels increases $24.5$ times.
    \item An $O(n^2)$ algorithm will have to run $600$ times faster.
    \end{itemize}
  \end{frame}

  \begin{frame}
    \frametitle{Why Better?}
    \begin{itemize}
    \item Again, remember that people are {\em greedy}.
    \item Remember faster, higher, stronger.
    \item We always seek possibility just beyond our capability.
    \item We want to live long and prosper (show a Vulcan gesture).
    \end{itemize}
  \end{frame}

  \begin{frame}
    \frametitle{Live Long and Prosper}
    \centerline{\pgfimage[width=0.6\textwidth]{Spock_performing_Vulcan_salute.jpg}}
    \footnote{\url{http://vignette4.wikia.nocookie.net/memoryalpha/images/5/52/Spock_performing_Vulcan_salute.jpg/revision/latest?cb=20090320072701&path-prefix=en}}
  \end{frame}

  \begin{frame}
    \frametitle{The Weather}
    \begin{itemize}
    \item The weather forecast resolution is proportional to the number of
      cells in the simulation.
    \item Taiwan has a length of 394 km and a width of 144 km.
    \item The area to simulate is 56736 square km. 
    \item Assume that the simulation model is 10 km in height.
    \item If the resolution is one cubic km, we need 567360 cells.
    \item If we refine the resolution to 100 m, the number of cells will be
      567360000.
    \item An $O(n^2)$ algorithms will have to run $1000000$ times faster.
    \end{itemize}
  \end{frame}

  \begin{frame}
    \frametitle{Discussion} 
    \begin{itemize}
    \item A naive matrix multiplication algorithm multiply two matrices of
      size $n$ by $n$ in $O(n^3)$ time.  If we increase the size of the
      matrix $n$ by a factor of 10, the execution time of this naive
      algorithm will roughly be?
    \end{itemize}
  \end{frame}

  \section{Scientific Computation}

  \begin{frame}
    \frametitle{Scientific Process} 
    \begin{enumerate}
    \item Observation
      \begin{itemize}
      \item Why does the apple fall on my head?
      \end{itemize}
    \item Theory
      \begin{itemize}
      \item Every matter attracts every matter.
      \end{itemize}
    \item Experiment
      \begin{itemize}
      \item Cavendish's experiment to verify the theory.
      \end{itemize}
    \end{enumerate}
  \end{frame}

  \begin{frame}
    \frametitle{Newton and Apple}
    \centerline{\pgfimage[width=0.7\textwidth]{newton-apple.png}}\footnote{\url{http://www.yalcafruittrees.com.au/wp-content/uploads/Isaac-Newton.png}}
  \end{frame}

  \begin{frame}
    \frametitle{Newton's Law of Universal Gravitation} Newton's law of
    universal gravitation states that any two bodies in the universe
    attract each other with a force that is directly proportional to the
    product of their masses and inversely proportional to the square of
    the distance between them.\footnote{\url{http://en.wikipedia.org/wiki/Newton\%27s_law_of_universal_gravitation}}
  \end{frame}

  \begin{frame}
    \frametitle{Formulation}
    \centerline{\pgfimage[width=0.6\textwidth]{universal-gravatation.pdf}}\footnote{NewtonsLawOfUniversalGravitation by I, Dennis Nilsson. Licensed under CC BY 3.0 via Wikimedia Commons \url{http://upload.wikimedia.org/wikipedia/commons/thumb/0/0e/NewtonsLawOfUniversalGravitation.svg/200px-NewtonsLawOfUniversalGravitation.svg.png}}
  \end{frame}

  \begin{frame}
    \frametitle{Cavendish's Experiment}
    \centerline{\pgfimage[width=0.6\textheight]{Cavendish_Torsion_Balance_Diagram.pdf}}\footnote{Cavendish Torsion Balance Diagram by Chris Burks (Chetvorno). Licensed under Public Domain via Wikimedia Commons \url{http://commons.wikimedia.org/wiki/File:Cavendish_Torsion_Balance_Diagram.svg\#mediaviewer/File:Cavendish_Torsion_Balance_Diagram.svg}}
  \end{frame}


  \begin{frame}
    \frametitle{Scientific Process} 
    \begin{enumerate}
    \item Observation
      \begin{itemize}
      \item Why does the galaxy have spirals?
      \end{itemize}
    \item Theory
      \begin{itemize}
      \item Every matter attracts every matter.
      \end{itemize}
    \item Experiment
      \begin{itemize}
      \item What are you talking about?
      \end{itemize}
    \end{enumerate}
  \end{frame}


  \begin{frame}
    \frametitle{Spiral Galaxy}
    \centerline{\pgfimage[width=0.6\textwidth]{spiral-galaxy.jpg}}\footnote{\url{http://i.space.com/images/i/000/022/667/wS4/spiral-galaxy-ngc1232-1600.jpg}}
  \end{frame}


  \begin{frame}
    \frametitle{Formulation}
    \centerline{\pgfimage[width=0.6\textwidth]{universal-gravatation.pdf}}\footnote{NewtonsLawOfUniversalGravitation by I, Dennis Nilsson. Licensed under CC BY 3.0 via Wikimedia Commons \url{http://upload.wikimedia.org/wikipedia/commons/thumb/0/0e/NewtonsLawOfUniversalGravitation.svg/200px-NewtonsLawOfUniversalGravitation.svg.png}}
  \end{frame}

  \begin{frame}
    \frametitle{Experiment} \Huge There are no ways we can put stars in a
    laboratory and observe the effects of gravity to incur spirals.
  \end{frame}

  \begin{frame}
    \frametitle{Discussion} 
    \begin{itemize}
    \item Give an example of scientific process, including observation,
      theory, and experiment.
    \item Give an example of experiment that cannot be conducted in a
      laboratory.
    \end{itemize}
  \end{frame}

  \begin{frame}
    \frametitle{Difficulties} 
    \begin{itemize}
    \item Experiments are {\em expensive} -- the stars may be expensive to
      buy.
    \item Experiments are {\em dangerous} -- you do not want to have a
      black-hole in your laboratory.
    \item Experiments are {\em unfeasible} -- my lab is not large enough.
    \item Experiments are {\em time consuming} -- I do not have billions
      of years for observation.
    \end{itemize}
  \end{frame}

  \begin{frame}
    \frametitle{Computer Simulation} 
    \begin{itemize}
    \item Simulation is cheap.
    \item Simulation is safe.
    \item Simulation is feasible.
    \item However, simulation is {\em slow}.
    \end{itemize}
  \end{frame}

  \begin{frame}
    \frametitle{How to be Better?}
    \Huge Having more than one CPU to work on the problem seems to be a
    reasonable choice.
  \end{frame}


  \begin{frame}
    \frametitle{Discussion} 
    \begin{itemize}
    \item Give an example of scientific simulation.
    \end{itemize}
  \end{frame}

  \begin{frame}
    \frametitle{Grand Challenge} A grand challenge is a fundamental problem in science or engineering, with broad applications, whose solution would be possible by the application of high-performance computing resources that could become available in the near future.\footnote{\url{http://en.wikipedia.org/wiki/Grand_Challenge}}
  \end{frame}

  \begin{frame}
    \frametitle{Grand Challenge Examples} 
    \begin{itemize}
    \item Computational fluid dynamics
    \item Electronic structure calculations
    \item Plasma dynamics for fusion energy technology and for safe and
      efficient military technology
    \item Calculations to understand the fundamental nature of matter,
      including quantum chromodynamics and condensed matter theory
    \item Symbolic computations
    \end{itemize}
  \end{frame}

  \begin{frame}
    \frametitle{How to be Better?}
    \Huge Having more than one CPU to work on the problem seems to be a
    reasonable choice.
  \end{frame}

  \begin{frame}
    \frametitle{Discussion} 
    \begin{itemize}
    \item Give an example of grand challenge problem.
    \end{itemize}
  \end{frame}


  \section{Definitions}

  \begin{frame}
    \frametitle{Parallel Computing}  \Huge Use multiple CPU's to solve a
    problem faster and/or better.
  \end{frame}

  \begin{frame}
    \frametitle{Why Parallel Computing?}
    \begin{itemize}
    \item We want computation to be faster and better.
    \item The performance of a single computer is limited, so the only way
      is to have {\em more} computers.
    \item We cannot find single-core CPU
      anymore\footnote{\url{http://www.intel.com/pressroom/kits/quickreffam.htm}},
      so it is essential to know how to get performance from a parallel
      computer.
    \end{itemize}
  \end{frame}

  \begin{frame}
    \frametitle{Limits}
    \begin{quote}
      There are technological and physical limits to the uni-processor
      performance that cannot be overcome.  For example, clock times
      cannot be shorter than the response time of electronic circuits,
      which are limited by physical laws.\footnote{Pangfeng Liu, The
        Parallel Implementation of N-body Algorithm,
        Ph.D. dissertation, 1994.}
    \end{quote}
  \end{frame}

  \begin{frame}
    \frametitle{Moore's Law}
    ``Moore's law'' is the observation that, over the history of computing
    hardware, the number of transistors in a dense integrated circuit
    doubles approximately every two
    years.\footnote{\url{http://en.wikipedia.org/wiki/Moore\%27s_law}}
  \end{frame}

  \begin{frame}
    \frametitle{An Illustration}
    \centerline{\pgfimage[width=0.6\textwidth]{Moore2020.png}}
  \end{frame}

  \begin{frame}
    \frametitle{Fifty Years of Moore's Law}
    \begin{itemize}
    \item Moore's Law is a direct consequence of the incredible and
      unique scaling heuristics of semiconductor manufacturing: by
      holding the cost per unit area of manufacturing
      constant\footnote{Chris A. Mack, IEEE Transactions on
        Semiconductor Manufacturing, Vol. 24, NO. 2, MAY 2011.
        \url{http://ieeexplore.ieee.org/stamp/stamp.jsp?arnumber=5696765}}
    \item The economic benefits of Moore's Law come from shrinking the
      transistor.
    \end{itemize}
  \end{frame}

\begin{frame}
\frametitle{Many More Years Left}
   \begin{itemize}
    \item Moore's Law is a learning curve by plotting minimum feature
      size as a function of cumulative revenue or area of silicon
      produced by the industry on a log-log scale.
    \item Moore's Law has kept on a relatively constant learning curve
      until about 2000.
    \item The acceleration of this Moore's learning curve over the
      last decade is likely an unsustainable, momentum-driven
      attempt to recapture past revenue growth rates.
    \item The industry, and the world, has enjoyed 50 great years of
      Moore's Law. There are unlikely to be many more years
      left.
   \end{itemize}
\end{frame}

  \begin{frame}
    \frametitle{Discussion} 
    \begin{itemize}
    \item Give at least three data points to support the Moore's Law.
    \end{itemize}
  \end{frame}


  \begin{frame}
    \frametitle{Parallel Computer} 
    \begin{itemize}
    \item A parallel computer is a system that has multiple processing
      units and supports parallel computing by {\em parallel programming}.
      \begin{itemize}
      \item Multicore
      \item Multiprocessor
      \item Multicomputer
      \end{itemize}
    \end{itemize}
  \end{frame}

  % multicore

  \subsection{Multicore CPU}

  \begin{frame}
    \frametitle{Multicore CPU}
    \begin{itemize}
    \item A multicore CPU has multiple cores as processing units.
    \item The cores share the memory and have usually have their cache.
    \item A memory arbitrator guarantees the consistency of shared memory
      and cache.
    \end{itemize}
  \end{frame}

  \begin{frame}
    \frametitle{Fujitsu A64FX}
    \begin{itemize}
    \item The A64FX is a 64-bit ARM architecture microprocessor
      designed by Fujitsu.
    \item Fujitsu collaborated with ARM to develop the processor; it is
      the first processor to use the ARMv8.2-A Scalable Vector Extension
      SIMD instruction set with 512-bit vector implementation.
    \item Each A64FX processor has 4 NUMA nodes, with each NUMA node
      having 12 compute cores, for a total of 48 cores per processor.
    \item Fujitsu designed the A64FX for the Fugaku.  More details later.
    \end{itemize}
  \end{frame}



  \begin{frame}
    \frametitle{Discussion} 
    \begin{itemize}
    \item Describe the number of cores in at least three current CPU's.
    \end{itemize}
  \end{frame}


  \subsection{Multiprocessor}

  \begin{frame}
    \frametitle{Multiprocessor} 
    \begin{itemize}
    \item A parallel system consists of {\em multiple} processors.
    \item Note that we usually do not distinguish processor and cores
      in this course, so we do not make a clear distinction between
      multiprocessor and multicore.
    \item The processors communicated with each other by reading and writing and shared memory, like a bulletin board.
    \end{itemize}
  \end{frame}

  \begin{frame}
    \frametitle{Dual Socket Server}
    \begin{itemize}
    \item This pircure has a GigaByte Dual socket motherboard.
      \footnote{\url{https://www.anandtech.com/show/16734/computex-2021-gigabyte-server-updates-mz72-hb0-for-dual-socket-3rd-gen-epyc}}
    \end{itemize}
    \centerline{\pgfimage[width=0.5\textwidth]{dualsocket.jpg}}
  \end{frame}




  \begin{frame}
    \frametitle{Top500}
    \begin{itemize}
    \item The TOP500 project ranks and details the 500 most powerful
      non-distributed computer systems in the world.
    \item The project started in 1993 and publishes an updated list of
      the supercomputers twice a year.
    \item The project aims to provide a reliable basis for tracking
      and detecting trends in high-performance computing and bases
      rankings on HPL.
    \item HPL is a portable implementation of the high-performance
      LINPACK benchmark implemented in Fortran for distributed-memory
      computers.
    \end{itemize}
  \end{frame}

  \begin{frame}
    \frametitle{November 2021}
    \begin{itemize}
    \item Fugaku continues to hold the No. 1 position that it first
      earned in June 2020.
    \item The Microsoft Azure system called Voyager-EUS2 was the only
      machine to shake up the top spots, claiming No. 10. Based on an
      AMD EPYC processor with 48 cores and 2.45GHz working together
      with an NVIDIA A100 GPU and 80 GB of memory,
    \item \url{https://www.top500.org/lists/top500/2021/11/}.
    \end{itemize}
  \end{frame}

  \begin{frame}
    \frametitle{Discussion} 
    \begin{itemize}
    \item Describe the number of processors within a node from any system
      in the top 10 of the top 500 list.
    \end{itemize}
  \end{frame}


  \subsection{Multicomputer}

  \begin{frame}
    \frametitle{Multicomputer} 
    \begin{itemize}
    \item A parallel system consists of {\em multiple} computers.
    \item The computers are connected by a {\em communication network}.
    \item Since the computers are independent and do not share memory,
      they communicate with each other by {\em messages}, like making
      phone calls.
    \end{itemize}
  \end{frame}

  \begin{frame}
    \frametitle{Multicomputer Examples} 
    \begin{itemize}
    \item Cluster computing
    \item Massively parallel computing
    \item Grid computing
    \item Cloud computing
    \end{itemize}
  \end{frame}



  \begin{frame}
    \frametitle{Cluster}
    \begin{itemize}
    \item A computer cluster consists of a set of loosely or tightly
      connected computers that work together so that, in many respects,
      they can be viewed as a single
      system.\footnote{\url{http://en.wikipedia.org/wiki/Computer_cluster}}
    \item Usually the computers are {\em loosely connected}. i.e., they
      are {\em not} connected by fast and expensive network.
    \item An economical alternative to those who cannot afford expensive
      parallel computers.
    \end{itemize}
  \end{frame}

  \begin{frame}
    \frametitle{My Definition}
    \begin{itemize}
    \item In my humble opinion, any computer system connected by a
      network, but not a shared memory, is a cluster.
    \item The point is that they do not have shared memory, so they
      can only communicate with the network.
    \item The network is not necessarily slow -- some networks are
      extremely fast, so the term {\em loosely coupled} may not be
      true in all cases.
    \end{itemize}
  \end{frame}

  \begin{frame}
    \frametitle{Supercomputer Fugaku}
    \begin{itemize}
    \item The total number of nodes in Fugaku is 158,976.
    \item A single CPU is a node, and two nodes on a board is a CPU Memory Unit (CMU).
    \item A bunch of blades (BoB) has Eight CMUs.
    \item A shelf has three BoBs.
    \item A computer rack has eight shelves.
    \item Fugaku is made up of 432 racks, of which 396 racks have 384 nodes, and 36 racks have 192 nodes.
    \item \url{https://www.r-ccs.riken.jp/en/fugaku/project}
    \end{itemize}
  \end{frame}

  \begin{frame}
    \frametitle{Supercomputer Fugaku}
    \begin{itemize}
    \item It is at RIKEN Center for Computational Science, Japan.
    \item The total number of cores is 7,630,848.
    \item The amount of memory is 5,087,232 GB.
    \item Linpack Performance (Rmax) is 442,010 TFlop/s.
    \end{itemize}
  \end{frame}

\begin{frame}
    \frametitle{Hierarchy}
    \begin{itemize}
    \item The processor is an A64FX 48C runninng at 2.2GHz, which is a multicore processor.
    \item Fugaku is a multicomputer connnnected by Tofu interconnnection network.
    \item Again, we do not intend to have a clear distinction between a
      multicore CPU and a multiprocessor.
    \item More details in the ``Architecture'' lecture.
    \end{itemize}
  \end{frame}

  \begin{frame}
    \frametitle{Discussion} 
    \begin{itemize}
    \item Describe the total number of cores within a node from any
      system in the top 10 of the top 500 list.
    \end{itemize}
  \end{frame}


  \section{Parallelism}

  \begin{frame}
    \frametitle{Parallelism} 
    \begin{itemize}
    \item Instruction-level parallelism
    \item Data parallelism
    \item Task parallelism
    \end{itemize}
  \end{frame}

  \begin{frame}
    \frametitle{Instruction-level Parallelism} 
    \begin{itemize}
    \item Instructions can be re-ordered and combined into groups which
      are then executed in parallel without changing the result of the
      program.
    \end{itemize}
  \end{frame}

  \begin{frame}
    \frametitle{Data Parallelism}
    \begin{itemize}
    \item Data parallelism is a form of parallelization of computing
      across multiple processors in parallel computing environments by
      distributing the {\em tasks} across different parallel computing
      nodes.
    \item Often in the form of loops.
    \end{itemize}
  \end{frame}

  \begin{frame}
    \frametitle{Data Parallelism Example}
    \programlisting{loop}{Data Parallelism}
    \begin{itemize}
    \item All the assignment can be done in parallel.
    \end{itemize}
  \end{frame}

  \begin{frame}
    \frametitle{Task Parallelism} 
    \begin{itemize}
    \item Task parallelism is a form of parallelization of computing
      across multiple processors in parallel computing environments by
      distributing the {\em tasks} across different parallel computing
      nodes.
    \item Also called functional parallelism.  Often in the form of
      function calls.
    \end{itemize}
  \end{frame}

  \begin{frame}
    \frametitle{Data Parallelism Example}
    \programlisting{task}{Task Parallelism}
    %% \begin{itemize}
    %% \item All the tasks can be done in parallel.
    %% \end{itemize}
  \end{frame}


  \begin{frame}
    \frametitle{Discussion} 
    \begin{itemize}
    \item Give an example of data parallelism and task parallelism.
    \end{itemize}
  \end{frame}


  \begin{frame}
    \frametitle{Dependency Graph}
    \begin{itemize}
    \item Every node is a task.
    \item Every edge is a dependency. This dependency is related to data or synchronization.
    \item A task can only start when all tasks that precede it finish, i.e., the tasks proceed in topological sort order.
    \end{itemize}
  \end{frame}

  \begin{frame}
    \frametitle{Dependency Graph}
    \begin{itemize}
    \item We can eat dinner only after we cook it. 
    \item We can listen to music only after turning on the radio.
    \item We can go to sleep only after dinner and music.
    \item We can listen to music while having dinner.
    \end{itemize}
  \end{frame}

  \begin{frame}
    \frametitle{Wavefront}
    \begin{itemize}
    \item Task parallelism must respect the dependency in the dependency graph.
    \item One can imagine that task parallelism is a series of wavefronts in the dependency graph.  
    \item We want to {\em increase} the number of tasks per wavefront to {\em reduce} the number of the wavefront. That is, we want to increase the {\em parallelism} of our algorithm so that it takes less time to complete.
    \end{itemize}
  \end{frame}

  \begin{frame}
    \frametitle{Discussion} 
    \begin{itemize}
    \item Draw the dependency graph of the previous task parallel example. You may add a {\em start} node and an {\em end} node to indicate the beginning and the end of execution. 
      Also, point out the wavefront and dependency in your drawing.
    \end{itemize}
  \end{frame}


  \begin{frame}
    \frametitle{Quantities} 
    \begin{description}
    \item [$P$] The number of processors
    \item [$N$] The number of tasks 
    \item [$L$] The longest path in the dependency graph
    \item [$W$] The maximum number of tasks in a wavefront
    \item [$T$] The execution time, assuming that it takes one unit of
      time to finish a task.
    \end{description}
  \end{frame}

  \begin{frame}
    \frametitle{Execution Time} 
    \begin{equation}
      N \geq T \ge \max(L, {N \over {\min(P, W)}})
    \end{equation}
    \begin{itemize}
    \item Increase $P$.
    \item Increase $W$, i.e., parallelism.
    \item Decrease $L$.
    \end{itemize}
  \end{frame}

  \begin{frame}
    \frametitle{Discussion} 
    \begin{itemize}
    \item Explain the inequality in the previous page.
    \end{itemize}
  \end{frame}

  \begin{frame}
    \frametitle{Parallel Programming} 
    \begin{itemize}
    \item Program a parallel system to perform parallel computing.
    \item That is why we are here.
    \item more about this in the ``Programming Model'' lecture.
    \end{itemize}
  \end{frame}

  \begin{frame}
    \frametitle{Automatic Parallel Programming?} 
    \begin{itemize}
    \item Can a program convert our sequential programs into parallel
      programs automatically?
    \item No, otherwise we will not be here.
    \item Computer scientists have been trying to invent {\em smart} compiler that automatically does it, but only have limited success.
    \item More details in the Programming Model lecture.
    \end{itemize}
  \end{frame}

  \section{Distributed Computing}

  \begin{frame}
    \begin{itemize}
    \item It is very often that people talk about parallel and distributed
      computing.
    \item The name of my laboratory is the {\em Laboratory of Parallel and Distributed Computing}.
    \item Many conferences and journals have both in their titles.
      \begin{itemize}
      \item IEEE International Parallel and Distributed Processing Symposium
      \item IEEE International Conference on Parallel and Distributed Systems
      \item Journal of Parallel and Distributed Computing
      \item IEEE Transactions on Parallel and Distributed Systems
      \end{itemize}
    \end{itemize}
  \end{frame}

  \begin{frame}
    \frametitle{Difference}
    \begin{description}
    \item[Parallel Computing] A set of processing unit to work on a job
      {\em at the same time}, i.e., the focus is that the computations is
      {\em temporarily} in parallel.
    \item[Distributed Computing] A set of processing unit to work on a job
      {\em at different locations}, i.e., the focus is that the
      computations is {\em geographically} distributed.
    \end{description}
  \end{frame}

  \begin{frame}
    \frametitle{Difference}
    \begin{itemize}
    \item The focus of parallel processing is {\em performance}. 
      We care very much about the speed that we can finish a job.
    \item The focus of distributed processing is {\em reliability}. We care very much about can we finish a job no matter what happens -- network down, hardware failure, earthquake, Godzilla attacks, etc.
    \end{itemize}
  \end{frame}

  \begin{frame}
    \frametitle{Race-car}
    \centerline{\pgfimage[width=0.6\textwidth]{racecar.jpg}}
    \footnote{\url{http://upload.wikimedia.org/wikipedia/commons/4/4a/Formel3_racing_car_amk.jpg}}
  \end{frame}

  \begin{frame}
    \frametitle{Tank}
    \centerline{\pgfimage[width=0.6\textwidth]{Leopard_2_A7.jpg}}
    \footnote{
      By Boevaya Mashina - Own work, CC BY-SA 4.0, https://commons.wikimedia.org/w/index.php?curid=43925342
    }
  \end{frame}

  \begin{frame}
    \frametitle{In this Course}
    \begin{itemize}
    \item This course will focus on performance, so we mostly discuss
      parallel processing, and will briefly discuss distributed processing
      when necessary.
    \item Distributed computing has become increasingly popular because of
      cloud computing and big data processing.
    \item Nevertheless, the role of parallel processing is still crucial
      since the processing speed is still essential.
    \item The focus is to combine the speed of parallel processing and the
      reliability of distributed computing into a ``parallel and
      distributed system''.
    \end{itemize}
  \end{frame}

  \begin{frame}
    \frametitle{Armored Vehicle}
    \centerline{\pgfimage[width=0.6\textwidth]{Plasan_SandCat.jpg}}
    \footnote{By Dino246 at English Wikipedia - Transferred from en.wikipedia to Commons., Public Domain, https://commons.wikimedia.org/w/index.php?curid=3423910}
  \end{frame}

  \begin{frame}
    \frametitle{Discussion} 
    \begin{itemize}
    \item Explain and describe the difference between parallel and
      distributed computing.  Give examples to illustrate your points.
    \end{itemize}
  \end{frame}


\end{CJK}
\end{document}
